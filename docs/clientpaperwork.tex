During the requirements elicitation process, the clients submitted a hand-written document of how they envisioned the system. 

The first two pages contain instructions for how the user would interact with the system, as follows:
\begin{itemize}
\item{Open NCBI}
\item{Type Acc. \# [Accession Number, used to locate the required sequence] into search}
\item{Pull out section of sequence which contains relevant bit for PCR (~500 bases?)}
\end{itemize}
This leads on to a section about primer design, interwoven with diagrams to illustrate their intention. In this document, the plan follows that the user would choose from
a list of 6 potential primers for each strand of the sequence. If the primer chosen is unsuitable, the user will be presented with a clue to inform them why that particular
primer would not be suitable. 
The user would then be prompted to return to the NCBI website, search for the primer sequence using the website's specialised search engine, and if it is unique, 
the system would move on.
The last page contained all the basic rules involved with primer design, which would need to be checked against for the user's choice.