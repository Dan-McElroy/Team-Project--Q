
%%%%%%%%%%%%%%%%%%%%%%%%%%%%%%%%%%%%%%%%%%%%%%%%%%%%%%%%%%%%%%%%%%%%%%%%%%%%%%
% This is a template for constructing your project plan document, but
% also to show the use of the l3deliverable class. Use pdflatex and
% bibtex to process the file, creating a PDF file as output (there is
% no need to use dvips when using pdflatex).
%
% Several meta data commands have been implemented to collect
% information such as deliverable identifier, project name etc (see
% below the \date command.

\documentclass{l3deliverable}

%%%%%%%%%%%%%%%%%%%%%%%%%%%%%%%%%%%%%%%%%%%%%%%%%%%%%%%%%%%%%%%%%%%%%%%%%%%%%%
% You can use the svn-multi package to automatically insert version
% control information into your document (an example of how to do this
% is shown below).  Make sure to set the 'svn:keywords' subversion
% property to 'Id' for the source file, for example, type:
%
% svn propset svn:keywords 'Id' d1.tex
%
% in the same directory as your 'd2.tex' file. 
%
% The information between the two $$ will now be updated when you next
% commit the file to your SVN repository.
%
% You can of course, just use this field to insert manual version
% information, e.g. 1.2, 1.2.1 ... instead.

\version{Version 2.3 \

Made \today \ by Ross Eric Barnie}

%%%%%%%%%%%%%%%%%%%%%%%%%%%%%%%%%%%%%%%%%%%%%%%%%%%%%%%%%%%%%%%%%%%%%%%%%%%%%%

\usepackage{url}
%\usepackage{glossaries}
\makeglossaries

\newacronym{psd}{PSD3}{Professional Software Development 3}
\newacronym{psd3}{PSD3}{Professional Software Development 3}

\newglossaryentry{psdGloss}
{
  name=Professional Software Development 3,
  description={Compulsory course for level 3 Computer Science and
    Software Engineering students at University of Glasgow. Designed
    to teach students how to develop software in a professional manner
    following professional standards.}
} remove comment if you ever fix the glossaries thing

%%%%%%%%%%%%%%%%%%%%%%%%%%%%%%%%%%%%%%%%%%%%%%%%%%%%%%%%%%%%%%%%%%%%%%%%%%%%%%
%% Check these macro values for appropriateness for your own document.

\title{Team Organisation}

%%authors
\author{
  Ross Eric Barnie \\
  Dmitrijs Jonins \\
  Daniel McElroy \\
  Murray Ross \\
  Ross Taylor
}

%%release date 
\date{\today}

\deliverableID{D1}
\project{Team Project 3}
\team{Q}

%%%%%%%%%%%%%%%%%%%%%%%%%%%%%%%%%%%%%%%%%%%%%%%%%%%%%%%%%%%%%%%%%%%%%%%%%%%%%%

\begin{document}

%%%%%%%%%%%%%%%%%%%%%%%%%%%%%%%%%%%%%%%%%%%%%%%%%%%%%%%%%%%%%%%%%%%%%%%%%%%%%%

\maketitle

%%%%%%%%%%%%%%%%%%%%%%%%%%%%%%%%%%%%%%%%%%%%%%%%%%%%%%%%%%%%%%%%%%%%%%%%%%%%%%
%% Standard section for all documents

\section{Introduction}

\subsection{Identification}

This document outlines the management plan of Team Q.

\subsection{Related Documentation}

General Project Information:
\begin{itemize}
\item{\url{http://fims.moodle.gla.ac.uk/file.php/129/tp3_2012_dates.pdf}}
\end{itemize}

Project Proposals (see reference 1895):
\begin{itemize}
\item{\url{http://www.dcs.gla.ac.uk/~jsinger/l3_props_2012.html}}
\end{itemize}

\subsection{Purpose and Description of Document}

This document will outline the role(s) of each member of the group, how those
members communicate with eachother and with the client, and will outline the
organisational risks specific to our group.

\subsection{Document Status and Schedule}

Currently this document is in draft and is scheduled to be released by 15th
November, with a review being carried out beforehand.

This document will be updated iteratively so when something changes which
contradicts this document, the document will be updated at the earliest
opportunity.
Pending review, the updated document will be released in the following days.

%%%%%%%%%%%%%%%%%%%%%%%%%%%%%%%%%%%%%%%%%%%%%%%%%%%%%%%%%%%%%%%%%%%%%%%%%%%%%%

\section{Roles}

\begin{description}
  \item[Project Manager] Dmitrijs Jonins
  \item[Customer Liaison and Quality Assuror] Ross Taylor
  \item[Chief Architect] Daniel McElroy
  \item[Test Manager] Murray Ross
  \item[Toolsmith and Librarian] Ross Barnie
\end{description}

%%%%%%%%%%%%%%%%%%%%%%%%%%%%%%%%%%%%%%%%%%%%%%%%%%%%%%%%%%%%%%%%%%%%%%%%%%%%%%

\section{Authority}

All decisions will be based on democratic vote, with roles being
assigned using the sports model referenced in the first lecture of PSD.

%%%%%%%%%%%%%%%%%%%%%%%%%%%%%%%%%%%%%%%%%%%%%%%%%%%%%%%%%%%%%%%%%%%%%%%%%%%%%%

\section{Communication}

The group have organised meetings at the following times:

\begin{itemize}
\item{Monday, 1200}
\item{Wednesday, 0900, with client and supervisor}
\item{Friday, 1500 or 1600 depending on timetable}
\end{itemize}

Other forms of communication have been established including our own
private Facebook group, which was created as the most convenient method
of communication due to the fact we all have a Facebook account prior to
this project and are familiar with its operation.
 
We have also created an online repository on GitHub to provide us with
a project management tool and version control.
It was decided that this was favourable to trac (used in PSD) as it is easier
to use, has better documentation, and is available online, so each member can
access it easily, indpendent of their working platform, rather than
trac which must be accessed via ssh tunnel to sibu.

%%%%%%%%%%%%%%%%%%%%%%%%%%%%%%%%%%%%%%%%%%%%%%%%%%%%%%%%%%%%%%%%%%%%%%%%%%%%%%

\section{Information Management}

Information will be kept in an online Git repository provided by GitHub,
including all documentation and implementation. The master fork is
held at \url{https://github.com/Dan-McElroy/Team-Project--Q}.

%%%%%%%%%%%%%%%%%%%%%%%%%%%%%%%%%%%%%%%%%%%%%%%%%%%%%%%%%%%%%%%%%%%%%%%%%%%%%%

\section{Organisational Risks}

Since we are assigning roles mostly on an as-and-when-required basis, there is
a risk that a role may be needed later which we have not defined or assigned.
Additionally, if someone were to face difficulties in their role and not 
communicate those difficulties to the rest of the group, that role could be 
overlooked or produce unsatisfactory results.

With a leader in the group who has ultimate authority on assignment of roles,
there could manifest resentment of the leader for assigning a member of the 
group to a role they do not enjoy or feel they are capable of doing.

%%%%%%%%%%%%%%%%%%%%%%%%%%%%%%%%%%%%%%%%%%%%%%%%%%%%%%%%%%%%%%%%%%%%%%%%%%%%%%

\appendix

\section{Glossary}
\begin{description}

\item[TP or TP3] {Team Project 3, a compulsory Level 3 Computing
    Science module where students are required to produce some piece
    of software. In our case, this system.}

\item[GitHub] {Refers to the website \url{github.com} which
    hosts the git repository for our project's documentation and
    implentation at \url{https://github.com/Dan-McElroy/Team-Project--Q}}

\item[Git] {A version control system used by Team Q to keep
    track of all digital files related to the system. Not to be
    confused with the term of `endearment' in the English language.}

\item[Portability] {A system which is portable is able to be used on a
    different computer to the one on which it was developed with no,
    or minimal, changes.}

\item[pdf] {A file format used to display documents.}

\item[LaTeX] {A document mark-up language used for all project documentation.}

\item[PCR] {Polymerase chain reaction, a biochemical technology used to amplify pieces of DNA by several orders of magnitude, creating copies of a particular DNA sequence.}

\item[Primer] {A strand that serves as a starting point for DNA synthesis.}

\item[Java] {An object-oriented computer programming language.}

\item[IDE] {An integrated development environment.}

\item[Netbeans] {An IDE for developing primarily in Java.}

\item[Eclipse] {An IDE for developing primarily in Java.}

\item[Swing] {The primary Java GUI widget toolkit.}

\item[JavaFX] {A software platform for creating applications.}

\item[GUI] {Graphical User Interface, an interface which a user interacts with using images.}

\item [PC] {Personal Computer, the terminals which the application will be run on.}

\item [DNA] {Deoxyribonucleic acid, a molecule that contains the genetic instructions used in the functioning of living organisms and many viruses.}

\item [Base] {The a,t,g and c symbols that comprimise a DNA sequence.}

\item [Nucleotide] {Biological molecules that form the building blocks of nucleic acids.}

\item [NCBI] {National Center for Biotechnology Information, a branch of the National Institudes of Health. From their website, users retrieve DNA sequences.}
\end{description}


%\section{Glossary 2}

%\printglossaries

\end{document}

%%%%%%%%%%%%%%%%%%%%%%%%%%%%%%%%%%%%%%%%%%%%%%%%%%%%%%%%%%%%%%%%%%%%%%%%%%%%%%
