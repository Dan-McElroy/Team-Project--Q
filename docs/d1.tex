
%%%%%%%%%%%%%%%%%%%%%%%%%%%%%%%%%%%%%%%%%%%%%%%%%%%%%%%%%%%%%%%%%%%%%%%%%%%%%%
% This is a template for constructing your project plan document, but
% also to show the use of the l3deliverable class. Use pdflatex and
% bibtex to process the file, creating a PDF file as output (there is
% no need to use dvips when using pdflatex).
%
% Several meta data commands have been implemented to collect
% information such as deliverable identifier, project name etc (see
% below the \date command.

\documentclass{l3deliverable}

%%%%%%%%%%%%%%%%%%%%%%%%%%%%%%%%%%%%%%%%%%%%%%%%%%%%%%%%%%%%%%%%%%%%%%%%%%%%%%
% You can use the svn-multi package to automatically insert version
% control information into your document (an example of how to do this
% is shown below).  Make sure to set the 'svn:keywords' subversion
% property to 'Id' for the source file, for example, type:
%
% svn propset svn:keywords 'Id' d1.tex
%
% in the same directory as your 'd2.tex' file. 
%
% The information between the two $$ will now be updated when you next
% commit the file to your SVN repository.
%
% You can of course, just use this field to insert manual version
% information, e.g. 1.2, 1.2.1 ... instead.

%\usepackage{svn-multi}
%\svnid{$Id: d1.tex 8 2012-09-26 18:29:40Z ross $}
%\version{SVN Revision \svnrev~ \

%Made \svnday/\svnmonth/\svnyear~ by \svnauthor}

%%%%%%%%%%%%%%%%%%%%%%%%%%%%%%%%%%%%%%%%%%%%%%%%%%%%%%%%%%%%%%%%%%%%%%%%%%%%%%

\usepackage{url}

%%%%%%%%%%%%%%%%%%%%%%%%%%%%%%%%%%%%%%%%%%%%%%%%%%%%%%%%%%%%%%%%%%%%%%%%%%%%%%
%% Check these macro values for appropriateness for your own document.

\title{Team Organisation}

%%authors
\author{
  Ross Eric Barnie \\
  Dmitrijs Jonins \\
  Daniel McElroy \\
  Murray Ross \\
  Ross Thomson
}

%%release date 
\date{27 September 2012}

\deliverableID{D1}
\project{PSD3 Group Exercise 1}
\team{Q}

%%%%%%%%%%%%%%%%%%%%%%%%%%%%%%%%%%%%%%%%%%%%%%%%%%%%%%%%%%%%%%%%%%%%%%%%%%%%%%

\begin{document}

%%%%%%%%%%%%%%%%%%%%%%%%%%%%%%%%%%%%%%%%%%%%%%%%%%%%%%%%%%%%%%%%%%%%%%%%%%%%%%

\maketitle

%%%%%%%%%%%%%%%%%%%%%%%%%%%%%%%%%%%%%%%%%%%%%%%%%%%%%%%%%%%%%%%%%%%%%%%%%%%%%%
%% Standard section for all documents

\section{Introduction}

\subsection{Identification}

This is the Management Plan of the Level 3 Project of Team Q.

\subsection{Related Documentation}

E.g.,
\begin{list}{}{}
\item PSD3 Group Exercise Description \
  
  \url{http://fims.moodle.gla.ac.uk/...}
\item (We weren't particularly sure what was to go here...)
\end{list}
 

\subsection{Purpose and Description of Document}
Describe the purpose, scope, and major functions of this document.

\subsection{Document Status and Schedule}

Not sure what to put here, particularly because we haven't met our supervisor 
at time of writing.

Describe the status, including goals and dates, for production and
revision of the document.  Documentation is often generated
incrementally and iteratively. If this is the case for this document,
also summarise here the planned updates and their release dates.


%%%%%%%%%%%%%%%%%%%%%%%%%%%%%%%%%%%%%%%%%%%%%%%%%%%%%%%%%%%%%%%%%%%%%%%%%%%%%%

\section{Roles}

\begin{description}
  \item[Project Manager] Dmitrijs Jonins
  \item[Customer Liaison] Ross Taylor
  \item[Librarian] Daniel McElroy
  \item[Test Manager] Murray Ross
  \item[Toolsmith] Ross Barnie
\end{description}

%%%%%%%%%%%%%%%%%%%%%%%%%%%%%%%%%%%%%%%%%%%%%%%%%%%%%%%%%%%%%%%%%%%%%%%%%%%%%%

\section{Authority}

Ultimate authority lies with Dmitrijs Jonins, however all decisions will be 
based on democratic vote, as in the Sports Model of authority referenced in
lecture one of Professional Software Development 3.

%%%%%%%%%%%%%%%%%%%%%%%%%%%%%%%%%%%%%%%%%%%%%%%%%%%%%%%%%%%%%%%%%%%%%%%%%%%%%%

\section{Communication}

Where and when group meetings will take place is still to be decided at the 
time of producing this document.

Other forms of communication have been established including our own
private Facebook group, which was created as the most convenient method
of communication due to the fact we all have a Facebook account prior to
this project and are familiar with its operation. 
We have also created a space on workflowy.com to keep track of current tasks 
needing to be done, including the production of this document, and whose 
responsibility it is to undertake those tasks.

%%%%%%%%%%%%%%%%%%%%%%%%%%%%%%%%%%%%%%%%%%%%%%%%%%%%%%%%%%%%%%%%%%%%%%%%%%%%%%

\section{Information Management}

Information will be kept in an online SVN repository provided by GitHub, and 
additional information will be made available via the previously mentioned 
Workflowy list.

%%%%%%%%%%%%%%%%%%%%%%%%%%%%%%%%%%%%%%%%%%%%%%%%%%%%%%%%%%%%%%%%%%%%%%%%%%%%%%

\section{Organisational Risks}

Since we are assigning roles mostly on an as-and-when-required basis, there is
a risk that a role may be needed later which we have not defined or assigned.
Additionally, if someone were to face difficulties in their role and not communicate
those difficulties to the rest of the group, that role could be overlooked or be
of poor standard.
With a leader in the group who has ultimate authority on assignment of roles,
there could manifest resentment of the leader for assigning a member of the group
to a role they do not enjoy or feel they are capable of doing.

%%%%%%%%%%%%%%%%%%%%%%%%%%%%%%%%%%%%%%%%%%%%%%%%%%%%%%%%%%%%%%%%%%%%%%%%%%%%%%

\appendix

\section{Glossary}

Nothing to add here yet but leaving in for future reference.

Including expansions of non-standard abbreviations and acronyms and
other key definitions.  You may find it useful to maintain a glossary
as a shared section amongst all your PSD documents. using the
\verb!\input{}! macro.

\section{Another appendix}

Again, nothing to add here but leaving section in for future reference.

Any relevant associated documentation, e.g., a meeting plan.

\end{document}

%%%%%%%%%%%%%%%%%%%%%%%%%%%%%%%%%%%%%%%%%%%%%%%%%%%%%%%%%%%%%%%%%%%%%%%%%%%%%%
