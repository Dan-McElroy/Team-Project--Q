\documentclass[a4paper]{article}
\title{First Demonstration - Feedback}
\usepackage{fullpage}
\begin{document}
\maketitle
\section{Evaluation Questions}
\begin{enumerate}
	\item{Was it always clear what to do to progress in the application? If it wasn't at any stage, please explain your issues:}
	\begin{itemize}
		\item{Yes - but lab book needs pictures interwoven with instructions.}
		\item{Yes - can click ''run'' instead of save.}
		\item{No right-click menu for ''paste'' - Ctrl-V must be used. Pressing 'next' instead of return confused me for a
				second - hitting ''return'' after the ''to'' box to progress to the next page might be an idea.}
		\item{The various stages are well-described. It was easy to understand the various transitions. It would have been
				helpful to have the figures beside the instructions.}
		\item{It was very clear what I have to do.}
		\item{The progress is clear.}	
		\item{Always clear :)}
		\item{Yes, the program was easy to use.}
		\item{Yes.}
	\end{itemize}
	
	\item{Aesthetically, did the application meet your demands and expectations? How could the visual aspect be improved?}
	\begin{itemize}
		\item{It could, but it was okay.}
		\item{Visually - could have a few images eg. DNA double helix.}
		\item{It was easy to understand and it was beneficial to see the different strands, complementary otherwise.}
		\item{Probably not be able to type into the sequence box as by accident you can write or delete nucleotides. It would
				be more easily to copy and paste with the mouse.}
		\item{Same colour. Instructions should be bolder and larger.}
		\item{While there was a clear layout, it would be handy if the pass  \& fail could remain open while reworking the
				primer. And showing where exactly your own primers start and end.}
		\item{The design was simple and very clear. It could be improved by adding some colour, but it's not necessary.
				The simple design made it easy to use.}
		\item{Very 90s, but does it the job :)}
		\item{It isn't a fancy design but don't think it has to be.}
	\end{itemize}
	
	\item{Did you find the \textit{Primer Design Rules} box to be a helpful summary of these rules? How could it be changed to
			provide more assistance?}
	\begin{itemize}
		\item{Allow it to stay on screen while you are working.}
		\item{Yes.}
		\item{It was useful, but it could be helpful to have a lot more detail in what would be considered a fail.}
		\item{It was really useful in order to design a correct primer.}
		\item{Good summary - should make clear that these are for guidance only and not set rules that can't be broken.}
		\item{It would be helpful if the rules were not 'set in stone' and you could be in one or two degrees difference.}
		\item{I found it helpful; right now I don't remember if you had included the melting temperature equation in the rules
				box, but if not, it should be added.}
		\item{Yes, summarised the rules nicely.}
		\item{Yes.}
	\end{itemize}
	
	\item{If you used the primer rules provided at any time, did you find the information provided useful in re-attempting the
			problem, and were the rule infractions covered in enough detail?}
	\begin{itemize}
		\item{Rule infractions could be highlighted in the sequence and an override provided for small errors.}
		\item{Yes.}
		\item{Yes.}
		\item{When it came to fixing the primers when failing then the rules proved to be a bit insufficient in helping with the
				new design of the primer.}
		\item{Yes.}
		\item{Yes. Description of where primers specifically failed was very useful.}
		\item{PASS/FAIL is good.}
		\item{I was aided but being out by 1 degree I thought would be fine.}
	\end{itemize}
	
	\item{Now that you have completed the application, do you feel your understanding of PCR and primer design techniques has
			improved?}
	\begin{itemize}
		item{I think it would if this was my first time doing this.}
		\item{-Blank-}
		\item{Yes.}
		\item{No.}
		\item{Yes, much better especially at how to design a primer and choose the correct one.}
		\item{I don't feel like they have improved, since we have already had to study these things, but it really helps you to
				notice how slight changes in primer sequence can change the properties(e.g. temperature) quite a lot.}
		\item{Definitely a good tool for design assistance.}
	\end{itemize}
	
	\item{Did you run into any technical issues with the application? If so, please describe them.}
	\begin{itemize}
		\item{No.}
		\item{Yes - primers were at correct Tm (64  \& 62) but software told me they were not within 3\u2106 of each other.}
		\item{At one point, when viewing the double-stranded sequence, the first group of 10 starting at 1051 was the opposite colour of what they should 					be, i.e. the top was blue and the bottom was orange.}
		\item{No technical issues.}
		\item{Primer disappearing when passed, no right click copy  \& paste.}
		\item{No. :)}
		\item{My reverse primer vanished after it was approved... and I hadn't written it down.}
	\end{itemize}
	
	\item{If you have any other comments you wish to make about the application ro the demonstration, please enter them here.}
	\begin{itemize}
		\item{Generally good, simple program. Immediate feedback on primer design.}
		\item{Allow for imperfect primers to be designed - this often happens in the real world! Sometimes they still work.
				On ''primer selection'' page - no ''copy''  \& ''paste'' button - most will know Ctrl-C/V, but not all students!
				Add back button onto ''DNA Sequence Entry'' panel.}
		\item{If the primers that you've chosen were highlighted in the sequence, that would have been helpful. If the program
				could check the primers one at a time?}
		
	\end{itemize}
\end{enumerate}


\section{Handout Notes}
\begin{itemize}
	\item{''In order to search for a compatible sequence...'' not very clear about nucleotides.}
	\item{2.2, does it have to be saved? Can we just run?}
	\item{3.3, first paragraph: Not clear.}
	\item{3.4, it should tell you how many repeats of a single base that you are allowed.}
	\item{Could you put in ''close fail'' rather than fail. For example, if your melting temperature is just outside the range,
			it should give a ''close fail'' so you could choose to ignore it.}
	\item{Primers failed due to self-annealing in 4 places but gives no indication where this is happening and at what
			temperature they would separate.}
	\item{Should have ability to override rejection manually.}
	\item{Where primer is rejected should still remain highlighted as lose place - only want to modify slightly.}
	\item{Should tell you number of bases you have selected for primer.}
	\item{Instruction text should be larger and bold.}
	\item{Start button prominence.}
	\item{Should let paste woth right-click option.}
	\item{Shuld be able to highlight DNA sequence want to use.}
	\item{No back button on page 2 (DNA Seq Entry page)}
	\item{No copy \& paste function on ''primer selection'' panel.}
	\item{When click on double strand view ''Primer Selection'' panel the number don't move with the sequence \& sequence is not
			highlighted.}
	\item{Should have an override button if students think that one particular parameter can be allowed.}
	\item{''Primer Selection'' window - remind them primer sequence must be 5' -> 3' and the view they see of complementary
			sequence is 3' -> 5'.}
	\item{What happens if you can't find a suitable sequence? - Go with what you can get!}
	\item{Allow box to get bigger - special needs.}
	\item{When error boxes come up they need expanded.}
\end{itemize} 
	


\end{document}
		
