The 'Primer Checks' methods are implementations of the established rules
and guidelines which are used in the process of Primer Design (seen in
the preliminaries section) to evaluate the effectiveness of a given 
Primer when used in the PCR process.

As with many other aspects of the project, the primer checks were split
between the two members of the 'back-end' sub-team. As well as making
this task more manageable, this approach offered the added benefit of
limiting the researching of primer design rules to 2 members, who each
only had to learn how to apply about 4 methods each.

The complexity of these rules varied greatly in difficulty, from
trivial checks such as “Primers should end in a base 'g' or 'c'” to
challenging checks such as checking how likely it is that one end of a
primer  will anneal to the other. The majority of these methods fell
into the first category and were fairly straightforward to implement,
however,  implementing the more difficult methods posed a serious
challenge.

Firstly, none of the members of the team had any experience with PCR or
Primer Design prior to the start of the project so every rule had to be
thoroughly studied and understood before beginning to design the
methods. However, even after spending time learning how the design
rules and guidelines are used, some methods still proved problematic.

\subsubsection{Melting Temperature}
The melting temperature check was among the easiest to implement, as we
were given a standard formula for calculating the temperature at which a 
primer would melt (\texttt{2*sum(a, t) + 4*sum(a, t)}) and the desired 
range in which a primer should rest (between roughly 50\degree C and 
60\degree C).

\subsubsection{Uniqueness}
To check that the primer is correctly placed in the system, it is first checked
that the primer cannot be found in the wrong strand (ie. that the forward
primer cannot be found on the complementary strand). When this is confirmed,
it is then checked that the correct strand contains one and only one instance
of the primer, and then that this instance is positioned correctly, relative
to the area of the sequence to be replicated.

\subsubsection{Primer Content}
Many of the primer rules were trivial to implement in the system, given
the string representation of a primer. Such rules include ensuring the
length of the primer lies between 20 and 30 bases, checking the last base
of a primer and confirming that no base is repeated more than 4 times
in a row. 

The melting temperature was among the easiest to implement, having been
given a standard formula for calculating the temperature at which a primer
would melt (2*sum(\verb£a, t£) + 4*sum\verb£a, t£) and the desired range
in which a primer should rest.


