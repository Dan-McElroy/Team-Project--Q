The 'Primer Checks' methods are implementations of the established rules
and guidelines which are used in the process of Primer Design (seen in
Section \ref{intro:prelims}) to evaluate the effectiveness of a given 
Primer when used in the PCR process.

As with many other aspects of the project, the primer checks were split
between the two members of the 'back-end' sub-team. As well as making
this task more manageable, this approach offered the added benefit of
limiting the researching of primer design rules to two members, who each
only had to learn how to apply about four methods each.

The implementation of these rules varied widely in difficulty, from
trivial checks such as ``Primers should end in a base `g' or `c''' to
more challenging problems such as checking how likely it is that a primer  
will self-anneal.

Firstly, as discussed in Section \ref{intro:motiv} none of the members of 
the team had any experience with PCR or Primer Design prior to the start of 
the project, and so every rule had to be thoroughly studied and understood 
before design of the methods could begin. However, even after spending time 
learning how the design rules and guidelines are used, some methods still 
proved problematic.

\subsubsection{Melting Temperature}
The melting temperature check was among the easiest to implement, as we
were given a standard formula for calculating the temperature at which a 
primer would melt (\texttt{2*sum(a, t) + 4*sum(a, t)}) and the desired 
range in which a primer should rest (between roughly 50\degree C and 
60\degree C).

\subsubsection{Uniqueness}
To check that the primer is correctly placed in the system, it is first checked
that the primer cannot be found in the other strand (e.g. that the forward
primer cannot be found on the complementary strand). When this is confirmed,
it is then checked that the correct strand contains one and only one instance
of the primer, and then that this instance is positioned correctly, relative
to the area of the sequence to be replicated.

\subsubsection{Primer Content}
Many of the primer rules were trivial to implement in the system, given
the string representation of a primer. Such rules include ensuring the
length of the primer lies between 20 and 30 bases, checking the last base
of a primer and confirming that no base is repeated more than 4 times
in a row. 

\subsubsection{Annealing}

As mentioned in Section \ref{intro:prelims}, there is an increased
chance of primer annealing happening if it is possible for a pair of
primers (or a single primer folding over on itself) to line up in such a
way as to produce 4 or more consecutive complementary bases between the
primers (or the two sections of a folded over primer). To aid with the
implementation of both the Self Annealing check and the Pair Annealing
check a function called \texttt{checkMatches()} was created. This function
takes in two primer subsequences (as strings) and returns the highest 
number of consecutive complementary bases in the two sequences.

\subsubsection{Self Annealing}

As described above, our \texttt{checkMatches} function compares
complementary bases between two strings. Therefore, to use this function
to check for annealing between two ends of a single primer the primer
has to be split into two strings the second of which must be reversed
before being sent to \texttt{checkMatches()} since during self annealing
the primer folds over as shown in section B of Figure 
\ref{fig:other:anneal}. To keep track of the index where the primer has
been split there is a variable, \texttt{split}.

The main body of the function is a while loop which calls
\texttt{checkMatches()} on the two sections of the primer which are
separated by \texttt{split} then increments \texttt{split} and if loops
again if \texttt{split} is at least 4 positions away from the end of the
primer. During this process the highest result from the
\texttt{checkMatches()} calls is recorded and this is used to determine
whether or not the primer is likely to self-anneal.

\subsubsection{Pair Annealing}

This function works in a similar fashion to the Self Anneal checking
method in that it compares the two primers (as opposed to two sections
of one primer in Self Annealing) in every possible alignment and stores
the highest number of matches.

The difference here is how the strings which are sent to the
\texttt{checkMatches()} function are selected in the
\texttt{pairAnneal()} function. The \texttt{pairAnneal()} function is
invoked on one \texttt{Primer} instance and the other primer is passed
as an argument. Firstly, the lengths of the primers are compared and the
string variables \texttt{max} and \texttt{min} are set appropriately (if
the primers are of the same length then which primer is set as
\texttt{max} and which is the \texttt{min} is irrelevant so the primer
which is passed as an argument is set as \texttt{min} and the primer
which the method was invoked on is set as \texttt{max}). Then, finding
the number of matches is done in three separate while loops. The first
while loop checks all possible alignments when the shorter primer
overlaps the longer one at the start, for example:
\begin{verbatim}
       agtcatcg
    acatca
\end{verbatim}
The second while loop checks the alignments when the shorter primer does
not overlap either end of the longer primer, for example:
\begin{verbatim}
     agatcgattgcagt
        agctaac
\end{verbatim}
And the third while loop checks the alignments when the shorter primer
overlaps the longer primer at the end:
\begin{verbatim}
     agtacgtaggtc
            tccagtac
\end{verbatim}
Once every possible alignment has been checked, the function uses the
highest number of matches to evaluate the likelyhood of the primers
annealing to each other.













