The 'Primer Checks' methods are implementations of the established rules
and guidelines which are used in the process of Primer Design to
evaluate the effectiveness of a given Primer when used in the PCR
process.
\\ \\
As with many other aspects of the project, the primer checks were split
between the two members of the 'back-end' sub-team. As well as making
this task more manageable, this approach offered the added benefit of
limiting the researching of primer design rules to 2 members, who each
only had to learn how to apply about 4 methods each.
\\ \\
The complexity of these rules varied greatly in difficulty, from
trivial checks such as “Primers should end in a base 'g' or 'c'” to
challenging checks such as checking how likely it is that one end of a
primer  will anneal to the other. The majority of these methods fell
into the first category and were fairly straightforward to implement,
however,  implementing the more difficult methods posed a serious
challenge.
\\ \\
Firstly, none of the members of the team had any experience with PCR or
Primer Design prior to the start of the project so every rule had to be
thoroughly studied and understood before beginning to design the
methods. However, even after spending time learning how the design
rules and guidelines are used, some methods still proved problematic.

***NEED SECTIONS FOR THE INTERESTING/DIFFICULT METHODS***
