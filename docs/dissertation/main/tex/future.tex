
With regards to future work on the project, we feel there are various improvements and tweaks that could be made to the application in order to improve it. These changes can be made in the future, in order to make it a better and more complete teaching tool for the Biology department. Also with regards to the feedback (See: Appendix \ref{app:questionnaireResponses}), we will continue to monitor any feedback from the students for any bugs we may need to fix, or any improvements they suggest.

\subsection{Future Improvements}

\begin{itemize}

\item Adding a highlighting primer function on the double stranded screen.

\item Add a function that ensures that if the user enters a sequence of capital letters into a primer box, that it will highlight it on the DNA sequence regardless.

\item Fixing inconsistencies on the resolution of each frame of the application.

\item Adding a 'hints' system for when users break rules, giving them tips on how to fix their primer to pass the specific rule.

\item Investigating the portability of the application further, as some feedback from a user (See: Appendix \ref{app:questionnaireResponses}) claimed that the application didn't run on a computer on campus (They say level 8, we presume they must mean a library computer).

\item Remaking the animation in flash, as it is a more suitable platform for static applications

\item Allowing the font size to be changed in the program to allow more accessibility.

\item Fixing the dynamic primer highlighting so that if two sequences that should be highlighted overlap each other, it displays both these sequences on the screen accordingly.

\item Increasing the usage of imagery in our program - logos, diagrams etc.

 

\end{itemize}











