With regards to future work on the project, we feel there are various improvements and tweaks that could be made to the application in order to improve it. These changes can be made in the future, in order to make it a better and more complete teaching tool for the School of Life Sciences.

Also with regards to the feedback (See: Appendix \ref{app:questionnaireResponses}), we will continue to monitor any feedback from the students for any bugs we may need to fix, or any improvements they suggest. Whilst the feedback we received has been mostly positive, it has highlighted areas in which the program can be modified and improved.

\subsection{Future Improvements}

Adding a highlighting primer function on the double stranded screen. This feature was never implemented due to the time constraints of the project. However, for the final application it would certainly improve usability for the user. It would also increase familiarity for the user, as they would see the same features implemented over each primer selection screen.

Whilst we did not envisage for this to be a problem, thanks to the feedback (See: Section \ref{eval:question}) provided it was clear that a user may try to enter their primer in using capital letters, for whatever reason. Therefore, a function that ensures that if the user enters a sequence of capital letters into a primer box, that it will highlight it on the DNA sequence regardless, should be implemented in the code for the program, to ensure this bug is fixed.

It has been noted that the standard resolution of the application begins at 800 x 600, but when the application reaches the animation panel it changes to 1024 x 768. A further improvement to be made therefore would be to fix the inconsistencies in the resolution, making it a standard size throughout usage.

We have discussed perhaps implementing a ``hints" system for primer design. This would attempt to guide the user when they enter an incorrect primer, perhaps suggesting some tips or methods to improve it, especially if it is close to passing the tests. The rules, however, are already provided for the user to view at their discretion however.

One issue that was raised in the questionnaire feedback (See: Appendix \ref{app:questionnaireResponses}) was that the animation would not run on a computer located on the Glasgow University campus (we are told the problem occurred on ``Level 8", which could refer to the Library, the Boyd Orr or any number of university buildings). Due to this, we have decided to investigate the portability of the application further, with particular regards to the animation sequence. Whilst we had tested for this beforehand more testing is required to find out the cause of this problem.

Another possible improvement to the application would be to remake the application in Flash \cite{Flash}. Flash is a global standard platform for static animations, and could possibly bring improvements to the look and feel of the application. 

The font size in the application is set to a standard size, and cannot be changed. An improvement that has been suggested to the implementation is that we allow the user to change the font size, to allow more accessibility.

A bug has been noted in further testing in the dynamic highlighting function. This occurs when a sequence is entered into the box, and due to the combination of the letters two sequences overlap each other, it will only highlight the first sequence on the chosen DNA. The proposed change to the function would highlight both these sequences, in different colours to indicate the difference.

Users have commented that the program looks very clinical and simple, keeping to a functional layout. Whilst this is a strong point of the program, it could perhaps be altered slightly with such alterations as images, different colour schemes, diagrams etc.

Finally, the team will make themselves available for general maintenance over the lifespan of the system.
