There are some terms that will be used later in the report that should be clarified now, so as to avoid confusion. The aim of the project is to creating a teaching tool for PCR, or Polymerase Chain Reactions. This is the process of amplifying a sequence of DNA thousands to millions of times. It should be explained that DNA sequences are made up of two strands, comprised of bases of the nucleotides Adenine, Thymine, Guanine and Cytosine, represented by the letters \verb£a£, \verb£t£, \verb£g£, and \verb£c£ respectively. Base pairing is when one base bonds with its complement on the other strand. \verb£a£ and \verb£t£ complement each other, and G and C complement each other.\\
Primers, used to select the sequence for PCR in a given selection of DNA, are shorter fragments of DNA, usually between 20 and 30 bases in length. For use in PCR, a primer must be chosen from the “left” of one strand (this is the forward primer) and the “right” of the other (this is the reverse primer), and these must obey a number of rules, which are the focus of the teaching tool:
\begin{itemize}
\item The primer must not self-anneal. This means that if the primer were to fold in on itself in any way that more than 3 bases in a row on one side paired to the base on the opposite side, the primer would fold over and become useless.
\item The melting temperature, calculated in degrees Celsius using a simple mathematical formula involving the frequency of \verb£a£s, \verb£t£s, \verb£g£s and \verb£c£s, must be between 50 and 65°C, and within 2-3°C of each other.
\item It must be unique within the strand.
\item The percentage of Gs and Cs within the sequence must be between 40\% and 60\%.
\item The length should be between 20 and 30 bases.
\item The same base should not be repeated several times in a row.
\item The last base of the primer should be either a \verb£g£ or a \verb£c£.
\item The primers should not anneal to each other. This means that the rule is broken if, at any point in the overlap of the primers, more than 3 bases in a row paired with the overlapping primer’s base.
\end{itemize}
It should also be explained that we are developing the teaching tool in Java, using Netbeans, a free IDE primarily designed to be used with Java, and developing the user interface with Swing, the primary Java GUI widget toolkit.
