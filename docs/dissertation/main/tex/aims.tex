The overall aim of this project is to produce a piece of software to
help Level 3 Life Sciences students taking the Molecular Methods
course (taught by Drs. Pamela Scott and Nicola Veitch) learn about PCR
and Primer Design Techniques and to allow them to test their knowledge
of these subjects. At the outset of the project, Scott and Veitch
helped us to separate this aim into key tasks to be completed and
important aspects of the interface design to be implemented:

\begin {enumerate}

\item The software should work as an interactive tutorial which users
  can work through. This requires:

\begin {itemize}
\item A number of areas for users to enter their own choice of data,
  such as a choice of DNA sequence to work with and the primers with
  which to operate on the selected strand. For this feature to be
  useful as an educational tool feedback must be provided upon data
  entry.
\item Users should be able to experiment with different data
  e.g. examining the different melting temperatures of different
  primers. This requires the ability to easily move forwards and
  backwards between the different stages of the tutorial.
\item To help newer users and students who are unfamiliar with PCR
  there should be simple instructions to guide users through the
  process and explain PCR throughout the application. There should
  also be a page displaying the rules of PCR and primer design which
  should be available at all times.
\end {itemize}

\item The software should be accessible to all users with a basic
  understanding of molecular biology, regardless of their different
  levels of knowledge, ability etc.:

\begin {itemize}
\item To achieve this, the interface should be uncomplicated and
  intuitive without compromising the required functionality. This will
  be aided by the instructions and help section mentioned above as
  well as labels placed next to any areas users can interact with.
\item Any section which makes use of colour should be designed with
  colour blind users in mind.
\end {itemize}

\item The software should improve upon the tools currently available
  for learning primer design. The main issues with these systems are: 

\begin {itemize}
\item The low level of interactivity offered by the systems, such as
  the numerous YouTube videos available on the subject
  \cite{youtube:taqExtension}. Users who are not actively working
  through a tutorial or a demonstration are likely to lose interest
  faster so it is important to make them involved with every step of
  the tutorial by having them design their own primers etc.
\item The available tools rarely go into detail about primer design
  specifically. One example of an interactive, well designed
  application that fails to convey the process of designing primers to
  a satisfactory degree is University of Utah's "PCR Virtual Lab"
  \cite{genScienceCenter2012}. Therefore, an important aim for the
  project is that primer design must be explained in detail and
  provide enough information to be informative, whilst remaining
  interesting to students using the system.
\end {itemize}

\item Another aim related to accessibility is that the users should be
  able to download and use the software from home. This means that the
  program must be able to run on a variety of different operating
  systems and computers with varying performance levels. With this in
  mind it was decided that the program should be written in Java due
  to it being highly portable.
\end {enumerate}
