The implementation of the graphical user interface (GUI) required a number
of decisions to be made before writing it could begin.
%---------------------------------------------------------------------
%---------------------------------------------------------------------

\subsection{GUI Framework}
\label{impl:ui:guiframework}

%---------------------------------------------------------------------

\subsubsection{Swing}
\label{impl:ui:guiframwork:swing}

Each member of the group had experience with the Swing framework,
though not all of it good.
The experience each member of the team had with Swing varied, and
although every member had agreed that they had not liked using it in
the past, we conceded that its integration with the Netbeans
Integrated Development Environment (IDE), discussed in section
\ref{impl:ui:ide:netbeans}, was extremely useful.

However, on investigating the framework more closely it was clear that
Swing was extremely well documented with full API specification
(\cite{swingAPI}), and in-depth tutorials (\cite{swingTutorial}).
This was a huge part of our decision as we felt that the documentation
provided would be more than adequate to allow us to use the framework
with relative comfort.

%---------------------------------------------------------------------

\subsubsection{JavaFX}
\label{impl:ui:guiframwork:javafx}

Another framework considered was JavaFX which no member of the team
had any experience with. 
Some members felt that this was a risk worth taking, given how much
they disliked Swing, discussed above.
In reality JavaFX was only briefly considered and totally disregarded
when, upon brief investigation, JavaFX was still a relatively new
framework, and was thus not quite as well documented as Swing,
particularly when it came to troubleshooting on online forums.

In addition, JavaFX required Java 7, which, again, no member of the
group had used before and which was not available, at the time, in the
Level 3 Laboratory where we would be working for the majority of the
year.
It seemed like too much of a risk to try to learn two different
technologies at the same time, while having to provide our own
development platforms, which, with various members of the team never
having used the Linux OS before, could potentially cause a number of
problems.

%---------------------------------------------------------------------

\subsubsection{Decision}
\label{impl:ui:guiframework:decision}

The investigation was carried out by the group's Toolsmith, Ross
Barnie, who presented the evidence discussed in the sections above
regarding the two frameworks to the rest of the team.
With this evidence the team voted in favor of using the Swing
framework with Java 6.

Retrospectively, Swing, and Java 6, are out-of-date technologies and
JavaFX is now packaged with Java 7 \cite{javafxOverview}, so the
application would have been more up-to-date or future-proof had we
used JavaFX.
Additionally, (some of) the computers in the level 3 lab now do have
Java 7 installed upon another project team requesting it, so our fears
over development platform problems were nullified, though this was
only after we had started development.

It was an unfortunate shortcoming of the research into JavaFX that the
group did not know about JavaFX's integration with the Netbeans IDE
which was seen as one of the key differences between the two
frameworks at the time of making the decision.

%---------------------------------------------------------------------
%---------------------------------------------------------------------

\subsection{IDE}
\label{impl:ui:ide}

One concern was that, in some members' experience, using two separate
IDEs was extremely time consuming, particularly while using version
control.
This was mostly due to various metadata that IDEs keep track of in
various files, however this meant that any small change to the source
code would change the metadata and therefore each commit would have to
involve adding it, which would be very time-consuming.

It is because of this experience that the group decided to work from a
single IDE, researched again by Ross Barnie.

%---------------------------------------------------------------------
\subsubsection{Netbeans}
\label{impl:ui:ide:netbeans}

Netbeans is an IDE which the team had had little experience with and had
only used in the context of building applications with GUIs created
using the Swing framework.
There was some trepidation to using Netbeans since most of the team
had associated their problems with Swing with Netbeans itself.
Upon further research, which involved using the IDE to build small
applications, Netbeans started much faster than Eclipse, discussed
below.
And the design interface was very simple and easy to use, with each
element being laid out the way you wish and the associated source code
being generated for you.
This meant that the design layout could be finished very quickly,
rather than spending our time writing hundreds of lines of source code
just for the interface.

In terms of Netbeans' metadata, it was quite minimal and would not
clutter the version control repository to an unacceptable degree.

%---------------------------------------------------------------------
\subsubsection{Eclipse}
\label{impl:ui:ide:eclipse}

The team had used Eclipse many times in the past, given that the
entirety of the coursework for level 2 was to be done using it.
Again, our experience of Eclipse is somewhat tainted by associations
with problems we faced at the time, such as a bug on the version for
Windows which meant that Eclipse would freeze if you tried to copy or
paste anything.

The team felt, unanimously, that Eclipse was slow, not only to
initially load, but also when trying to write code.
Editing-wise, Eclipse was rather cumbersome and not much better than a
text editor.
Also the requirement to bind the ``Workspace'' was seen as a potential
point for confusion and errors.

In addition, the team felt that the missing design interface seen on
Netbeans, discussed above, was a huge disadvantage and would cause a
significant loss of time, simply due to the volume of code we would
have to write instead of being auto-generated.

Members of the team also pointed out that Eclipse has a tendency to
create a large amount of metadata which would clutter the version
control repository.

%---------------------------------------------------------------------
\subsubsection{No IDE}
\label{impl:ui:ide:noide}

It was briefly considered to have no IDE at all and simply use text
editors.
This would allow for extremely fast editing in a very comfortable
environment, since most text editors, such as Vim or Emacs, are highly
customisable and can launch in a matter of seconds.
Text editors would also not require metadata, keeping our version
controlled directories clean.

However, the obvious problem with no IDE is that troubleshooting
problems becomes very tedious very quickly, and unlike IDEs, you
cannot automatically import a missing package or method, nor can there
be any auto-generated code for that matter.

%---------------------------------------------------------------------
\subsubsection{Decision}
\label{impl:ui:ide:decision}

When the evidence above was given to the team, we were also discussing
which GUI Framework to use (as discussed in section
\ref{impl:ui:guiframework}) and it became obvious that integration
with the framework would be key to helping us develop the GUI.

We therefore decided to work with the Netbeans IDE because of the
design interface, minimal metadata, and lack of (known) bugs that
would affect us in any meaningful way.

Retrospectively, this was the correct decision.
Even if we had chosen a different GUI framework, the advantages of the
easy-to-edit design interface far outweigh any problems we had with
it.
 
