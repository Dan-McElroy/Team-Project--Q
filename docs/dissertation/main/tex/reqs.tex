% WILL NEED THE FOLLOWING:
% FIGURES - original meeting notes - FIND THEM
%         - Dmitrijs' designs, and any others we can find

\subsubsection{Inital Requirements Gathering}
% The sheet we were presented with Meeting 1 - flow of application more 
% or less already decided
% Molecular Methods lab book
% Links to Youtube videos and Utah thang
The requirements gathering process for the application began immediately.
At the first meeting, our clients presented us with a document outlining
what it was that they wanted from the end product, including a very
early step-by-step walkthrough of the application they envisioned. This
proved to be a key tool in bringing us up to speed with what the %thing
should accomplish, and really sped up the initial requirements gathering
phase. Obviously, this design was altered and adapted throughout the
project, but the steps served to provide a rough guidline that we
followed throughout development. The aim of the project, as described in
the aforementioned document, are as follows:
\begin{quotation}
To design a PCR-primer design exercise to complement teaching of a
Molecular Methods course to Level 3 Life Sciences Undergraduates. This
exercise will be integrated into a new part of the lab which we are
designing based around diagnosis of HIV using PCR. You will need to
understand the theory behind PCR and primer design in order to achieve
this.
\end{quotation}
This statement alone is helpful as it tells us about our userbase, where
and how the application will be used, and what background knowledge is
required in order to thoroughly understand the premise of the project.

On the subject of background knowledge, along with this document, we 
were given the Molecular Methods lab book, in order to see how Primer 
Design is currently taught in the course and get a better idea of how
it worked ourselves.
% TALK ABOUT MOL METHODS BOOK, WHAT COULD BE IMPROVED.
%---------------------------------------------------------------------------------------------DON'T FORGET ABOUT THIS BROSSOLINI

Finally, within the first two or three meetings we were sent links to
various multimedia teaching tools for Primer Design, as described in
Section \ref{intro:currentSystems}. Along with our own research, this
gave us an informed view of what else is out there, the positives and
negatives of these current approaches, and what we could improve upon
in our own product.

From these first few weeks of meetings with the clients, we drafted a
requirements document, which was presented to the clients and agreed upon,
and presented the following requirements:

\t\textbf{System Scope}
\begin{itemize}
\item{The main aim of this system is to act as a teaching tool to aid 
students in learning how to design primers for PCR experiments and 
should be usable in a teaching environment or by people on their home 
computers.}
\item{It should function as an interactive, step-by-step guide through 
the process of PCR on a DNA sequence of the users choice. The user is 
required to access the NCBI website and copy and paste their choice
of DNA sequence into the system. The system should provide feedback if 
the user enters incorrect primers. The system should then check if the 
melting temperatures of the primers are in the required range. The user
is then given a link to perform primer blast to check if the primers 
they have chosen are unique.}
\item{The system should also provide the user help with completing each 
task by providing relevant rules for each task and giving the user 
instructions about how to use websites and resources outwith the system 
(NCBI, primer blast etc.).}
\item{When the user has provided an appropriate pair of primers the 
system will then show an animation of the PCR reaction taking place.}
\end{itemize}
\t\textbf{Non-Functional Requirements}
\begin{itemize}
\item{The system is expected to be used at students’ homes or in the 
Biology lab computers, so portability is essential for the system to 
work to the clients’ expectations.}
\end{itemize}
We found that while several aspects of the system deviated from the
document we were handed in the first meeting, the requirements largely
stayed the same throughout.

\subsubsection{Design Feedback}
% Produced several versions of user interface mockups
Throughout the project, we maintained a weekly meeting schedule with our
supervisor and clients, and despite scheduling difficulties at least one
of the clients was present at every one of these meetings. This allowed
us the opportunity to improve our design iteratively through multiple
pitches, internalising the feedback given over the following week to 
produce a design more in line with their requirements.

Over the course of these meetings, the clients provided us with  

When the team had formed a solid idea of the layout and flow of the
system, we drew up some early mockups of the system's user interface
(discussed in further detail in Section \ref{design:ui}) and presented
them to the clients during 
one of the weekly meetings %DATE IF POSSIBLE
to get their feedback on the design and how they felt it met the 
requirements. 

We asked a number of questions related to both the user
interface and our understanding of PCR, and found that they were largely
satisfied with prototype design. This prototype did not feature the
animation described in the requirements, but we received valuable feedback
on the validity of the system described by the prototype as a teaching
tool.


\subsubsection{Implementation Feedback}
% At weekly meetings during implementation, gave repeated demonstrations
% Clients were always more than happy to provide a number of suggestions
% and improvements in keeping with the requirements of the product.
Once implementation was fully underway and the application had reached
a demonstrable state, we began presenting our progress at the weekly client
meetings. We received substantial feedback on how to change the UI and 
how the primer checks were handled in order
to best improve the product as a teaching tool, and to be as accurate to
the primer design process as possible.  
