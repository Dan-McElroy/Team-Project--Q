\section{Project Outline}
\paragraph{Successes}
\begin{itemize}
\item Technical state of the application. The version of the system that
  we submitted to the clients for evaluation by the students, while lacking
  some of the features we plan to implement, is fully featured, has been
  tested on all common operating systems and has only shown two minor problems.
  
\item Smooth requirements gathering process. Aided by the comprehensive
  preparation of the clients before our first meeting and their patience and
  willingness to answer any questions we had about the primer design process,
  as well as the consistent meeting schedule set in place with the supervisor
  and clients, the requirements gathering process was efficient and comprehensive.

\item Satisfaction of intended users. When we released the program to the
  students (who, despite having more experience with PCR than our target users,
  represented the userbase as accurately as possible),
  we found that the majority of users were pleased with the application. We also
  found that their understanding of primer design had been improved as a result
  of its use.

\item Despite the team's total lack of knowledge related to PCR and Primer Design,
  much less any field of biology, we were still able to create a successful teaching
  tool that has a demonstrable positive effect on students' learning. 

\item Submitted product met all requirements. Through our implementation process, we
  regularly consulted the requirements document drafted in the early weeks of the
  project to ensure that the application successfully met the requirements laid out
  therein, and we feel that this has been achieved. The data received from the
  questionnaire tells us that the system is effective as a teaching tool. 
  Justification, the application is designed as a step-by-step guide which incorporates
  the use of NCBI and provides feedback upon the entry of an incorrect primer,
  the substance of which we worked hard with the clients to ensure was informative
  and helpful to the user. The system includes a panel, accessible when required,
  which displays a lecturer-approved list of the primer design rules. 
  Lastly, thanks to its development in Java along with other design measures taken,
  the application is portable to all desktop computers.
\end{itemize}

\paragraph{Failures}
\begin{itemize}
\item Inconsistent group organisation. In the opening meetings, the group decided on
  a number of roles for each member. For example, Dmitrijs Jonins was named Project 
  Manager, and Murray Ross was given the role of Quality Assurance Manager. However,
  as the project progressed, we found that these roles were either inapproprate for
  that member, or simply unnecessary, and were gradually abandoned. Ideally, these
  decisions would have been made much earlier in the project and the group could have
  moved on, but no official reorganisation took place, and consequently the delegation
  of various tasks occasionally presented more of a problem than it should have.  

\item Implementation started late. The initial design stage of the project lasted until
  the beginning of December, and the team made plans to split the workload and begin work
  over the winter break. In reality, more could have been achieved in the weeks leading up
  to and during the break, and the lack of significant progress in the early stages of 
  implementation led to a disproportionate workload in the remainder of implementation.

\item Minor bugs remain. From the questionnaire feedback, we have been made aware of
  a number of small bugs that do not ultimately affect the educational aspect of the
  software, but present a problem for a subset of users nonetheless. We have made plans
  to eliminate these bugs in our future work. 

\end{itemize}

\section{Lessons Learned}
\paragraph{Development Skills}
Throughout the project, the team gained further experience with Java and Swing, improving
our confidence with the language and how it should be effectively utilised in the context
of larger projects. On a similar note, the team was introduced to version control software, 
and by the end of the project some members had grown more confident in their use of the 
more advanced techniques.
  
\paragraph{Production Skills}
As the first major project for each team member, we learned valuable lessons about setting
deadlines, timekeeping, role assignment, delegation and many other aspects of professional
software development.

\paragraph{Interpersonal Skills}
We found that the project was an excellent opportunity to learn about interpersonal relations 
in the context of a long-term production team, allowing us to improve our skills in delegation,
constructive criticism and general diplomacy. 

\paragraph{Client Relations}
As previously explained, the involvement of Dr. Scott and Dr. Veitch afforded us an excellent
opportunity to develop our skills in client relations, including maintaining constant 
communication, seeking regular feedback and managing expectations.  
