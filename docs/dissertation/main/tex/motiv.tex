When selecting a project at the outset of the course, we identified
several factors associated with this project that motivated us to take 
it on.

Chief among these factors was the aim of building an interactive teaching
tool. Some members of the team expressed an interest in going on to
create educational software after completion of their degree, and this
project would serve as ideal experience in developing such software.

Another part of the project that excited us was the opportunity of working
within the university to potentially improve the education of our peers.
Several members of the group have friends on the course in question, and
these friends have provided valuable feedback along with their colleagues.
Another aspect of the involvement of Drs. Scott and Veitch was in gaining
valuable experience in client relations. Several of the projects on offer,
while interesting, did not involve any stakeholders other than their 
supervisors, and so we felt this project would be a unique opportunity to
put into practice the lessons we had learned from other courses about
requirements gathering, without the risk associated with the involvement
of an external business entity, for whom the consequences of failure might
be more severe.

Lastly, the element of the project that excited us the most was the chance
to do work related to a field that we had absolutely minimal experience
with. The sum total of biology-related experience on the team was Ross 
Taylor's Higher qualification in Biology in secondary school, and that
placed us in a great position to learn about certain elements of
molecular biology from an outsider's perspective. 

\subsubsection{Current Systems}
In order to understand the motivation for the development of the
system, Drs. Scott and Veitch provided us with links to several systems 
currently in place which attempt to make learning this process more 
interactive and/or visual. However, videos and multimedia in general 
have been questioned as teaching aids in the past 
\citep{gamingRedefines2004}. As expressed in this paper, simply because 
the information is in video or multimedia format does not necessarily 
mean that it is benefiting the learning of its viewers, or creating the 
correct environment to encourage learning. Interactivity, along with 
other factors, are key to engaging people to learn.

The first was a video hosted on YouTube \citep{youtube:taqExtension},
made by demonstrators within the School of Life Sciences.
During its eighteen second duration, the video shows various elements
of the PCR process including change in temperature and the role of the
primer.
However, it was commented by the team and by the clients that it was
insubstantial in terms of information delivery, several of the stages
of PCR are omitted with no mention of primer design, and in terms
of interactivity.

Another video hosted on YouTube \citep{youtube:PCR}, currently referred
to on School of Life Sciences' website, is similar in style to a
lecture with slides and a voice-over which repeats the textual
information on each slide.
While this video is far more informative than the previous one, with
each stage of PCR clearly described, and with visually pleasing
animations, it lacks in explicit primer design and again in
interactivity.

Finally, an animation from the University of Utah, titled ``PCR
Virtual Lab'' \citep{genScienceCenter2012}.
This is a much more interactive experience and allows the user to use
virtual pipettes in order to simulate what you would do in a lab
situation when performing PCR.
Additionally, the information it provides, while slightly basic in the
beginning for our target users, is extensive and very informative to
the novice user, such as Biology-illiterate Computing Scientists.
While this is a much more interactive and, compared to the
alternatives described above, much more informative experience, it
fails to provide the user with the theoretical background information,
particularly on primer design (required to fully understand the
process and why the reaction occurs), and does not allow the user to
test their ability to select good primers, the most difficult aspect
of PCR.																									% FIND A REFERENCE FOR THIS 
