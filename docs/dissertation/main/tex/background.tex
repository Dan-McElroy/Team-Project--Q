% To put across:
% Client names, idea behind it (mine?)
% 		BRIEF intro to PCR
% Cooperation and exchange of skills for the win
% Skills it allows us to expand upon

PCR, in the simplest terms, is used to amplify (reproduce) a specific region of a 
DNA strand. Initially developed in 1983 by Dr. Kary Mullis \cite{shortHistory}, 
PCR has since found widespread use in a number of areas of genetic analysis such as 
detection of infectious disease organisms \cite{hiv} and DNA profiling of suspects 
by forensic scientists as well as numerous research and medical applications. 
Given this wide range of applications, it is clear to see why PCR is an essential 
technique for students to learn and fully understand.

The idea for our system was first put forward by Dr Pamela Scott and Dr
Nicola Veitch, the clients for the project. Their involvement stems from 
their role in teaching the Molecular Methods course, compulsory for all 
Level 3 Life Sciences students. 

In their experience, they had found that a number of students struggle with the 
PCR element of the course, particularly the task of designing primers for the 
process (described in more detail in Section \ref{intro:prelims}), and decided 
that a teaching tool for this element of the course would be helpful in engaging 
the students toward the subject. 

They had seen a number of videos and an interactive web application on PCR, but these 
either did not focus on the actual design of primers or were severely limited by the 
lack of interactivity. It was with these missing elements in mind that they created 
the detailed problem specification that we were presented with at the beginning of 
the project.
