The idea for the system was first put forward by Dr Pamela Scott and Dr
Nicola Veitch, the clients for the project. 

Their involvement comes from their role in teaching the Molecular Methods 
course, compulsory for all Level 3 Life Sciences students. 

They had noticed that a number of students struggle with the PCR element of the 
course, particularly the task of designing primers for the process (described in
more detail in Section \ref{intro:prelims}), and decided that a teaching tool for 
this element of the course would be helpful in engaging the students toward the 
subject. 

They had seen a number of videos and an interactive web application on PCR, but these 
either did not focus on the actual design of primers or were severely limited by the 
lack of interactivity. 

It was with these missing elements in mind that they created the detailed problem 
specification that we were presented with at the beginning of the project.


%As previously explained in section 1.3, the project came from a dissatisfaction 
%from the teaching staff of Molecular Methods with the current method of teaching 
%PCR, but in order to better understand what it is the lecturers sought after, 
%research into PCR education systems currently in place became a necessity.

