\subsubsection{Dynamic Primer Highlighting}

Dynamic Primer Highlighting was suggested early on in the requirements
elicitation process by the clients as something that would be very
helpful to students studying primer design. The initial specification
for this feature was as follows:
	
“As the user enters their choice of primer in one of the boxes at the
top of the page, instances of the primer should be highlighted in
real-time in the corresponding box below containing the DNA sequence for
the strand on which this primer should appear.”

In a later client meeting the following addition was made to improve the
effectiveness of the feature:

“Primers should be highlighted in different colours to indicate their
suitability.”

This was a feature that all members of the team liked from a very early
stage because, even with our limited knowledge about primer design we
could see how this form of immediate feedback had the potential to
improve the usability of the system if it were to be implemented well.
Due to this level of popularity among both the clients and the team
members the feature was given high priority. However, since we knew it
would be complex to implement and would require calls to other planned
modules of the system, such as the primer checking functions, it was
also decided that this should be one of the last features to be
implement.

Ultimately, this feature turned out to be one of the most problematic
features to implement as it involved using features of Java which we had
no real experience with, specifically the Swing classes
'ChangeListener', 'Highlighter' and 'Painter' as well as integrating
other modules of our code to provide the required primer checking. This
level of difficulty meant that this feature actually took multiple
attempts to implement correctly.

***NEED SECTION ABOUT THE PREVIOUS ATTEMPTS***

***NEED SECTION ABOUT THE FINAL SOLUTION TO PROBLEM***
