%- 3 objects, Sequence, Primer, TestResult
%Basic in construction, all classes have toStrings and booleans.

\begin{figure}[h]
  \begin{center}
    \includegraphics[width=0.75\textwidth]{./images/currentBuild/modelClassDiagram.png}
    \caption{
      \label{fig:currentBuild:model}
      Model Class Diagram 
    }
  \end{center}
\end{figure}

The models used in the application are few in number and very simple, BLAH BLAH BLAH BLAH

As seen in Figure \ref{fig:currentBuild:model}, we used three classes to represent the data:
Primer, TestResult and Sequence.

\paragraph{Primer}
The Primer class is purely designed to test user-designed primers. It has 
one attribute, \texttt{code}, the String representing the user's primer
which is tested against in the primer test methods, which make up the 
remainder of the primer. The class is primarily made up of methods 
designed to test \texttt{code} against the various rules described in Section 
\ref{intro:prelims}. The method \texttt{test()} gathers runs all above test 
methods and returns a larger TestResult, indicating if the user's Primer 
is adequate outside of the larger context of the sequence.

\paragraph{TestResult}
TestResult is a class used to format the output of one or multiple primer 
tests. TestResult uses an enumerated type called PassState with values
\texttt{PASS}, \texttt{FAIL} and \texttt{CLOSEFAIL}, the last of which describes 
a state where the primer's value from a test lies outside of the recommended
values, but is close enough to a pass to be acceptable, provided this is 
only the state of a minority of tests. TestResult uses two ArrayLists, one of
PassStates (\texttt{passes}) and another of Strings (\texttt{out}), to keep track 
of the state and informative message to be displayed to the user for each 
test. 

Its methods are concerned with concatenating results into larger 
TestResults. \texttt{perfect()} will return true if all entries in 
\texttt{passes} equal a \texttt{PASS}. \texttt{adequate()}, the method that is 
checked to gate the user's access past the Primer Selection panel returns 
false if any of the tests returned 	exttt{FAIL}, or if more than 60\% 
returned \texttt{CLOSEFAIL}, and returns true otherwise.

\paragraph{Sequence}
This class contains two Strings, \texttt{oStrand} and \texttt{cStrand}, 
representing the strand of DNA that the user took in and the ''complementary''
strand that is generated when the sequence is constructed. Integers 
\texttt{start} and \texttt{end} represent the indexes of the start and end of 
the selected area in the sequence, and Primers \texttt{fPrimer} and 
\texttt{rPrimer} are, obviously, representations of the user's primers.
 
\texttt{parser()} is used for both sequence entry and primer input, by taking in
a String and returning a new String with all non-\texttt{atgc} characters removed.
\texttt{complement()} is a very simple function used throughout the application
that takes in one character representing a base, and returns its complement, i.e.
\texttt{complement('a')} would return \texttt{t}. \texttt{isUnique()} and \texttt{
tempDifference()} check the user's primers against the rules concerned with the
larger Sequence. \texttt{primerTest()} uses all other test methods to return a
TestResult in accordance with the  																										% phrasing?
