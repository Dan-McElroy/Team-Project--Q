In order to test the final system against the requirements set out in
section *REFERENCE REQUIREMENTS*, the team decided to produce a
questionnaire which would be given to students using the application
for them to give us feedback.
The raw data for this feedback is shown in appendix
\ref{app:questionnaireResponses}.

While other methods of testing exist (e.g. experimental) it was felt
that with the time constraints in place we would not be able to run a
test requiring our direct involvement and that we would receive more
data from a questionnaire.

%---------------------------------------------------------------------
\subsection{Questionnaire System}

Many questionnaire services exist on the internet, such as
SurveyMonkey \cite{surveyMonkey}, FreeOnlineSurveys
\cite{freeOnlineSurveys}, and Google Docs Forms
\cite{googleDocsForms}.

The team decided to use the Google Form, simply because most members
of the team have a Google account, and because it was free.

On closer inspection it could also export its responses to a Google
Docs Spreadsheet which itself could be exported to a variety of
formats.

%---------------------------------------------------------------------
\subsection{The Questions}

Generally, the questionnaire needed to be as concise as possible to
avoid any data being corrupted by frustration at the questionnaire.
To this end we kept the number of questions in general to a minimum
and only asked for a description if it was necessary.

\subsubsection{Skill Level}
At the demonstration, the team realised that not every user will be in
the expected age bracket of 18 to 20-years-old who have used computers
their entire lives.
This led to the decision that the feedback should take the user's age
into account, to ensure that there are no patterns in the data
suggesting a particular age group struggles with the application.

In addition, we ask the user for their ``Confidence'' with computers
in general, on a scale from 0 to 5 ie no middle option so there has
to be a bias to confident or not confident.
This completely subjective question should allow us to see if people
who perhaps do not use computers on a daily basis handle the
application, since we believe we made the application (and
accompanying user guide in appendix \ref{app:userGuideCurrent}) as
user-friendly as possible.

We are also aware of some colour-use in the program that could be
problamatic for colour-blind people so we ask the user if they are or
not.

To get an idea of the user's knowledge of primer design, we ask the
user for their self-assessed understanding before using the system, on
a 0 to 5 scale.

All of this data provides us with the baseline skill level with
computers and with primer design of the user, before they use the
system.

\subsubsection{User's Experience with the System}
The following questions were designed to provide us with feedback on
the user's experience with the system.

The first of these follows the self-assessed understanding of primer
design before using the system, with a self-assessed understanding
after using the system, on the same scale.
This will provide us with test data towards the requirement to teach
students about primer design and PCR.

Ideally, as discussed in section *REFERENCE REQUIREMENTS GATHERING*,
the system would be used as a revision tool in students' own time
and/or in a laboratory setting with tutors on hand to help the
student with any problems.
To see if we met this requirement, we asked the user if they would use
the system to study one of the following options:
\begin{itemize}
\item Both Primer Design and PCR
\item Just Primer Design
\item Just PCR
\item Neither
\end{itemize}
Which provides the user, and the team, with every possible answer to
the question, in the most concise way possible.


%---------------------------------------------------------------------
\subsection{Feedback Analysis}