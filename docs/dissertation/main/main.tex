\documentclass{l3proj}

\title{A Hands-on Approach to Learning Molecular Biology Techniques}
\author{
  Ross Eric Barnie \\
  Dmitrijs Jonins \\
  Daniel McElroy \\
  Murray Ross \\
  Ross Taylor
}
\date{18th March 2013}

\usepackage{mystyle}

\begin{document}
\maketitle
\begin{abstract}

PCR and Primer design is a concept taught to all students in the School of Life Sciences at Glasgow
University. This application has been designed to allow users to reinforce the concepts they have
learnt and improve their skills in these areas. A user enters a DNA sequence into the application,
before being asked to select a correct forward and reverse primer, and gives them feedback on these
selections. This is useful as it allows users to see where they are going wrong in their design, and
which areas they need to study. By using this as a teaching tool, we hope the School of Life Sciences
can enhance students understanding of these concepts.

\end{abstract}
\centerline{Acknowledgements}

First of all, we would like to thank our supervisor, Dr Gethin Norman, for his guidance, organisational skills and support throughout this project. We would also like to thank Dr Nicola Veitch and Dr Pamela Scott for their assistance with all matters relating to PCR and Primer design.

\educationalconsent
\tableofcontents

%=====================================================================
\chapter{Introduction}
\label{intro}

\section{Preliminaries}
\label{intro:prelims}
% Need to explain NCBI, BLAST

There are some terms that will be used later in the report that should
be clarified here, so as to avoid confusion.
The aim of the project is to creating a teaching tool for PCR, or
Polymerase Chain Reactions.
This is the process of amplifying a specified sequence of DNA thousands to
millions of times.
It should be explained that DNA sequences are made up of two strands,
comprised of bases of the nucleotides Adenine, Thymine, Guanine and
Cytosine, represented by the letters \verb£a£, \verb£t£, \verb£g£, and
\verb£c£ respectively.
Base pairing is when one base bonds with its complement on the other
strand.
The bases \verb£a£ and \verb£t£ complement each other, and bases
\verb£g£ and \verb£c£ complement each other.

Primers, used to select the sequence for PCR in a given selection of
DNA, are shorter fragments of DNA, usually between 20 and 30 bases in
length.
For use in PCR, a primer must be chosen from the ``left'' of one
strand (this is the forward primer) and the ``right'' of the other
(this is the reverse primer), and these must obey a number of rules,
which are the focus of the teaching tool:
\begin{itemize}
\item Neither primer should self-anneal. 
  This means that if the primer were to fold over
  in such a way that that more than 3 bases in a row on one side
  complemented the bases on the opposite side, this primer would pair 
  with itself and become useless for the purposes of PCR.
\item The melting temperature of each primer, calculated in degrees
  Celsius using a simple mathematical formula involving the frequency
  of \verb£a£s, \verb£t£s, \verb£g£s and \verb£c£s, should be between
  50 and 65\degree C, and within 2-3\degree C of each other.
\item The forward primer should be unique within the first strand, and
  not appear in the second, complementary strand. Likewise, the
  reverse primer should be unique within the second strand and should
  not appear in the first.
\item The percentage of \verb£g£s and \verb£c£s within each primer
  should be between 40\% and 60\%.
\item The length of each primer should be between 20 and 30 bases.
\item The same base should not be repeated several times in a row in
  either primer.
\item The last base of each primer should be either a \verb£g£ or a
  \verb£c£.
\item The primers should not anneal to each other. 
  This means that if the primers were put side by side, at no point of
  overlap should more than 3 bases in a row complement the overlapping
  primer’s base.
\end{itemize}

It should be noted that many of these rules are not precise, and do
not involve rigid limits for success or failure.
In fact, these ``rules'' more closely resemble rough guides.
Obviously, the nature of programming does not lend itself to ``rough
guides'', and so we followed advice from Pamela Scott and Nicola
Veitch on how best to structure these tests to effectively represent
the imprecision of their boundaries.

\begin{figure}[h]
  \begin{center}
    \includegraphics[width=0.75\textwidth]{./images/other/annealing.jpg}
    \caption{
      \label{fig:other:anneal}
      Annealing	
    }
  \end{center}
\end{figure}

The ends of strands of DNA are often referred to as the ``5'-end'' and
the ``3'end'' which are named as such due to the positioning of carbon
atoms near to the ends of the strand. This naming convention allows DNA
strand orientation to be described as either ``5'-3''' or ``3'-5'''.
This is relevant to PCR and primer design because strands anneal to
each other by the 5'-ends of the strands joining to the 3'-ends of the
other strand. This is shown in section A of Figure
\ref{fig:other:anneal}.










%--------------------------------------------------------------------------

Two external services that are integral to the successful use of the system
are the National Center for Biotechnology Information (NCBI) \cite{ncbi},
and the use of a Basic Local Alignment Search Tool. NCBI is, for the purposes
of this exercise, a vast database of DNA sequences and related information,
from which the user can obtain a sequence on which to perform PCR (though this
can also be obtained from other sources). At the end of the exercise, the
user is encouraged to use the BLAST functionality on the NCBI website to ensure
that their primer is specific to their intended PCR target.

It should also be explained that we developed the teaching tool
in Java, using Netbeans, a free IDE primarily designed to be used with
Java, and developing the user interface with Swing, the primary Java 6
GUI framework and that we stored all project-related files in a
version-controlled repository hosted by GitHub.



\section{Aims}
\label{intro:aims}
The overall aim of this project is to produce a piece of software to
help Level 3 Life Sciences students taking the Molecular Methods
course (taught by Drs. Pamela Scott and Nicola Veitch) learn about PCR
and Primer Design Techniques and to allow them to test their knowledge
of these subjects. At the outset of the project, Scott and Veitch
helped us to separate this aim into key tasks to be completed and
important aspects of the interface design to be implemented:

\begin {enumerate}

\item The software should work as an interactive tutorial which users
  can work through. This requires:

\begin {itemize}
\item A number of areas for users to enter their own choice of data,
  such as a choice of DNA sequence to work with and the primers with
  which to operate on the selected strand. For this feature to be
  useful as an educational tool feedback must be provided upon data
  entry.
\item Users should be able to experiment with different data
  e.g. examining the different melting temperatures of different
  primers. This requires the ability to easily move forwards and
  backwards between the different stages of the tutorial.
\item To help newer users and students who are unfamiliar with PCR
  there should be simple instructions to guide users through the
  process and explain PCR throughout the application. There should
  also be a page displaying the rules of PCR and primer design which
  should be available at all times.
\end {itemize}

\item The software should be accessible to all users with a basic
  understanding of molecular biology, regardless of their different
  levels of knowledge, ability etc.:

\begin {itemize}
\item To achieve this, the interface should be uncomplicated and
  intuitive without compromising the required functionality. This will
  be aided by the instructions and help section mentioned above as
  well as labels placed next to any areas users can interact with.
\item Any section which makes use of colour should be designed with
  colour blind users in mind.
\end {itemize}

\item The software should improve upon the tools currently available
  for learning primer design. The main issues with these systems are: 

\begin {itemize}
\item The low level of interactivity offered by the systems, such as
  the numerous YouTube videos available on the subject
  \cite{youtube:taqExtension}. Users who are not actively working
  through a tutorial or a demonstration are likely to lose interest
  faster so it is important to make them involved with every step of
  the tutorial by having them design their own primers etc.
\item The available tools rarely go into detail about primer design
  specifically. One example of an interactive, well designed
  application that fails to convey the process of designing primers to
  a satisfactory degree is University of Utah's "PCR Virtual Lab"
  \cite{genScienceCenter2012}. Therefore, an important aim for the
  project is that primer design must be explained in detail and
  provide enough information to be informative, whilst remaining
  interesting to students using the system.
\end {itemize}

\item Another aim related to accessibility is that the users should be
  able to download and use the software from home. This means that the
  program must be able to run on a variety of different operating
  systems and computers with varying performance levels. With this in
  mind it was decided that the program should be written in Java due
  to it being highly portable.
\end {enumerate}


\section{Background}
\label{intro:background}
% To put across:
% Client names, idea behind it (mine?)
% 		BRIEF intro to PCR
% Cooperation and exchange of skills for the win
% Skills it allows us to expand upon

PCR, in the simplest terms, is used to amplify (reproduce) a specific region of a 
DNA strand. Initially developed in 1983 by Dr. Kary Mullis \cite{shortHistory}, 
PCR has since found widespread use in a number of areas of genetic analysis such as 
detection of infectious disease organisms \cite{hiv} and DNA profiling of suspects 
by forensic scientists as well as numerous research and medical applications. 
Given this wide range of applications, it is clear to see why PCR is an essential 
technique for students to learn and fully understand.

The idea for our system was first put forward by Dr Pamela Scott and Dr
Nicola Veitch, the clients for the project. Their involvement stems from 
their role in teaching the Molecular Methods course, compulsory for all 
Level 3 Life Sciences students. 

In their experience, they had found that a number of students struggle with the 
PCR element of the course, particularly the task of designing primers for the 
process (described in more detail in Section \ref{intro:prelims}), and decided 
that a teaching tool for this element of the course would be helpful in engaging 
the students toward the subject. 

They had seen a number of videos and an interactive web application on PCR, but these 
either did not focus on the actual design of primers or were severely limited by the 
lack of interactivity. It was with these missing elements in mind that they created 
the detailed problem specification that we were presented with at the beginning of 
the project.


\section{Motivation}
\label{intro:motiv}
When selecting a project at the outset of the course, we identified
several factors associated with this project that motivated us to take 
it on.

Chief among these factors was the aim of building an interactive teaching
tool. Some members of the team expressed an interest in going on to
create educational software after completion of their degree, and this
project would serve as ideal experience in developing such software.

Another part of the project that excited us was the opportunity of working
within the university to potentially improve the education of our peers.
Several members of the group have friends on the course in question, and
these friends have provided valuable feedback along with their colleagues.
Another aspect of the involvement of Drs. Scott and Veitch was in gaining
valuable experience in client relations. Several of the projects on offer,
while interesting, did not involve any stakeholders other than their 
supervisors, and so we felt this project would be a unique opportunity to
put into practice the lessons we had learned from other courses about
requirements gathering, without the risk associated with the involvement
of an external business entity, for whom the consequences of failure might
be more severe.

Lastly, the element of the project that excited us the most was the chance
to do work related to a field that we had absolutely minimal experience
with. The sum total of biology-related experience on the team was Ross 
Taylor's Higher qualification in Biology in secondary school, and that
placed us in a great position to learn about certain elements of
molecular biology from an outsider's perspective. 

As previously explained in section \ref{intro:background}, the project
came from a dissatisfaction from the teaching staff of Molecular Methods
with the current method of teaching PCR, but in order to better understand
what it is the lecturers sought after, research into PCR education systems
currently in place became a necessity. 

\subsubsection{Current Systems}
\label{intro:currentSystems}
In order to understand the motivation for the development of the
system, Drs. Scott and Veitch provided us with links to several systems 
currently in place which attempt to make learning this process more 
interactive and/or visual. However, videos and multimedia in general 
have been questioned as teaching aids in the past 
\cite{gamingRedefines2004}. As expressed in this paper, simply because 
the information is in video or multimedia format does not necessarily 
mean that it is benefiting the learning of its viewers, or creating the 
correct environment to encourage learning. Interactivity, along with 
other factors, are key to engaging people to learn.

The first was a video hosted on YouTube \cite{youtube:taqExtension},
made by demonstrators within the School of Life Sciences.
During its eighteen second duration, the video shows various elements
of the PCR process including change in temperature and the role of the
primer.
However, it was commented by the team and by the clients that it was
insubstantial in terms of information delivery, several of the stages
of PCR are omitted with no mention of primer design, and in terms
of interactivity.

Another video hosted on YouTube \cite{youtube:PCR}, currently referred
to on School of Life Sciences' website, is similar in style to a
lecture with slides and a voice-over which repeats the textual
information on each slide.
While this video is far more informative than the previous one, with
each stage of PCR clearly described, and with visually pleasing
animations, it lacks in explicit primer design and again in
interactivity.

Finally, an animation from the University of Utah, titled ``PCR
Virtual Lab'' \cite{genScienceCenter2012}.
This is a much more interactive experience and allows the user to use
virtual pipettes in order to simulate what you would do in a lab
situation when performing PCR.
Additionally, the information it provides, while slightly basic in the
beginning for our target users, is extensive and very informative to
the novice user, such as Biology-illiterate Computing Scientists.
While this is a much more interactive and, compared to the
alternatives described above, much more informative experience, it
fails to provide the user with the theoretical background information,
particularly on primer design (required to fully understand the
process and why the reaction occurs), and does not allow the user to
test their ability to select good primers, the most difficult aspect
of PCR.																									% FIND A REFERENCE FOR THIS 


%=====================================================================
\chapter{Design}
\label{design}

\section{Requirements}
\label{design:reqs}
% WILL NEED THE FOLLOWING:
% FIGURES - original meeting notes - FIND THEM
%         - Dmitrijs' designs, and any others we can find

\subsubsection{Inital Requirements Gathering}
% The sheet we were presented with Meeting 1 - flow of application more 
% or less already decided
% Molecular Methods lab book
% Links to Youtube videos and Utah thang
The requirements gathering process for the application began immediately.
At the first meeting, our clients presented us with a document outlining
what it was that they wanted from the end product, including a very
early step-by-step walkthrough of the application they envisioned. This
proved to be a key tool in bringing us up to speed with what the %thing
should accomplish, and really sped up the initial requirements gathering
phase. Obviously, this design was altered and adapted throughout the
project, but the steps served to provide a rough guidline that we
followed throughout development.

Along with this document, we were given the Molecular Methods lab book,
in order to see how Primer Design is currently taught in the course.
% TALK ABOUT MOL METHODS BOOK, WHAT COULD BE IMPROVED.

Finally, within the first two or three meetings we were sent links to
various multimedia teaching tools for Primer Design, as described in
Section \ref{intro:currentSystems}. Along with our own research, this
gave us an informed view of what else is out there, the positives and
negatives of these current approaches, and what we could improve upon
in our own product.
 
% Probably more, dunno.

\subsubsection{Design Feedback}
% Weekly meetings, reaffirming theory behind PCR and primer design
% Produced several versions of user interface mockups
Throughout the project, we maintained a weekly meeting schedule with our
supervisor and clients, and despite scheduling difficulties at least one
of the clients was present at every one of these meetings. This allowed
us the opportunity to improve our design iteratively through multiple
pitches, internalising the feedback given over the following week to 
produce a design more in line with their requirements. 

\subsubsection{Implementation Feedback}
% At weekly meetings during implementation, gave repeated demonstrations
% Clients were always more than happy to provide a number of suggestions
% and improvements in keeping with the requirements of the product.


\section{UI}
\label{design:ui}
The User Interface design was developed based on the feedback and
guidelines set by the clients.
As it was an abstraction, we decided a rough mock-up was adequate for
flexibility of the design document.
Four main stages of the UI were separated into five slides which are
provided and explained below.


\subsubsection{Sequence entry (figure \ref{fig:UiDes:slide1})}
The first slide shows the sequence entry screen, where the user copies
the DNA sequence on which to perform PCR into the provided box.
The complementary strand is generated right away and shown in green
text, with each line being a complementary sequence of the one above.
The numbers on the left show the index number at the start of the
corresponding line, as a way for the user to quickly find the part of
the sequence to be amplified in the next stage and to provide better
orientation in the sequence altogether.
Scrolling through the sequence is done by pressing the arrow buttons
on the right or using scrolling with the mouse.
Once the user is content with the sequence entered, he can press the
"Go" button to advance to the next stage.

\begin{figure}[h]
  \begin{center}
	\includegraphics[width=0.6\textwidth]{./images/UiDes/Slide1.JPG}
    \caption{
      \label{fig:UiDes:slide1}
      Initial design, Sequence entry
    }
  \end{center}
\end{figure}

\subsubsection{Specification of target area (figure \ref{fig:UiDes:slide2})}
The second slide shows the selection of the DNA sequence part to be
amplified by PCR.
The user enters the first and the final base indices in the "From" and
"To" text fields respectively, and the chosen area is highlighted
after the user presses the "Enter" button.
All of the previously introduced functions of the UI, including DNA
sequence editing, are present at this stage also.
When the user is content with the area specified, he can press the
"Go" button to advance to the next stage.
The "Go" button color is changed to clearly differentiate from the
"Enter" button.

\begin{figure}[h]
  \begin{center}
	\includegraphics[width=0.6\textwidth]{./images/UiDes/Slide2.JPG}
    \caption{
      \label{fig:UiDes:slide2}
      Initial design,Specification of target area
    }
  \end{center}
\end{figure}

\subsubsection{Primer selection (figure \ref{fig:UiDes:slide3})}

The user is prompted to copy/paste or type the forward and the reverse
primers into their respective text fields located above the sequence.
At the time of the interface being designed the team was still
becoming accustomed to the terminology involved so the design
mistakenly shows ``Backward'' primer instead of the correct
``Reverse''.
The rules of primer design can be viewed by pressing the ``Rules''
button.
If the user decides to change the target area or the sequence itself,
he can go back to the previous stage by pressing the ``Back'' button. 

\begin{figure}[h]
  \begin{center}
	\includegraphics[width=0.6\textwidth]{./images/UiDes/slide3.jpg}
    \caption{
      \label{fig:UiDes:slide3}
      Initial design, Primer selection - initial screen
    }
  \end{center}
\end{figure}

When the user has entered the primers into their respective boxes, as
shown in figure \ref{fig:UiDes:slide3}, the user can press the ``Go'' button
to go to the next stage if the primers are correct, or be shown an
error message if they are not.

\begin{figure}[h]
  \begin{center}
	\includegraphics[width=0.6\textwidth]{./images/UiDes/Slide4.JPG}
    \caption{
      \label{fig:UiDes:slide4}
      Initial design, Primer selection - primers entered
    }
  \end{center}
\end{figure}

Each primer is checked for correctness according to the primer design
rules and the user is shown a message explaining why his chosen primer
is incorrect for the target sequence.
In the case that both primers are incorrect, a separate list of errors
is shown for each one.
The user can then press the white arrow button to go back to selecting
the primers.

\begin{figure}[h]
  \begin{center}
	\includegraphics[width=0.6\textwidth]{./images/UiDes/slide5.jpg}
    \caption{
      \label{fig:UiDes:slide5}
      Initial design, Primer selection - error message
    }
  \end{center}
\end{figure}

\subsubsection{Melting temperature check (figure \ref{fig:UiDes:slide6})}
The last slide shows the primers chosen by the user still in their
respective boxes and their melting temperatures just below.
Both temperatures must be in the range of 50 to 60 degrees Celcius for
PCR to work, so if they are not the user is suggested going back to
the previous screen and selecting a different primer pair.
As before, pressing ``Back'' will take the user to a previous stage.
The user is also advised to visit the NCBI website \cite{ncbi} and
performing primer blast on his selected primers to check them for
specificity. Lastly, pressing the "Go" button will open a new window
with an animation of PCR in action.

\begin{figure}[h]
  \begin{center}
	\includegraphics[width=0.6\textwidth]{./images/UiDes/Slide6.JPG}
    \caption{
      \label{fig:UiDes:slide6}
      Initial design, Primer melting temperatures
    }
  \end{center}
\end{figure}



%=====================================================================
\chapter{Implementation}
\label{chap:impl}

\section{Team Distribution}
\label{impl:teamdistribution}
Before the team began implementing the application, we decided to
split the team into three smaller sub-teams, in order to maximise the
use of everyone's time.
These groups were to:
\begin{itemize}
\item Design and implement data models and associated custom methods
\item Implement the graphical user interface
\item Design and implement an animation to show the process of PCR
\end{itemize}

The team's lead programmer, Daniel McElroy, took the lead on this
decision and, while noting each member's particular preferences,
decided to split the team in the following way:
\begin{description}
\item[GUI]{Ross Eric Barnie, Murray Ross}
\item[Data Models and Custom Methods] {Daniel McElroy, Ross Taylor}
\item[Animation] {Dmitrijs Jonins}
\end{description}

While it may have been unnecessary to assign a team-member entirely to
the animation, the team felt that Dmitrijs would work best on his own
and meant that the rest of the team could work as they had done up to
this point, as a team.


\section{Programming Language}
\label{impl:proglang}
When discussing implementation, the group briefly considered a small
number of languages before quickly settling on Java as our 
implementation language. This decision was made in light of our 
collective experience with the language as a consequence of the Java 
Programming course taken in the previous academic year
\cite{javaProgramming2}, and our knowledge of existing GUI frameworks 
that would suit our purposes, described below in Section \ref{impl:ui}.

Additionally, one of the requirements stated, in section
\ref{design:reqs}, that portability was essential and Java is
extremely portable with 1.1 billion desktop PCs \cite{aboutJava},
meaning that no matter what platform the students use, the application
should run.



\section{User Interface}
\label{impl:ui}
The implementation of the graphical user interface (GUI) required a number
of decisions to be made before writing it could begin, including of
course, how the team would be distributed between implementing the GUI
and the data models.

%---------------------------------------------------------------------

% Not the best place for this, but finding a better one will require
% restructuring during editing process.

\subsection{Programming Language}
\label{impl:ui:programminglanguage}

When discussing implementation, the group quickly settled on Java as
the language in which to implement the application, due to our 
collective experience with it as a consequence of the Java Programming
course taken in the previous academic year, and our knowledge of
existing GUI frameworks that would suit our purposes.

%---------------------------------------------------------------------

\subsection{GUI Framework}
\label{impl:ui:guiframework}

From a brief research period at the start of the implementation 
process, we settled on two possible options for a GUI framework to use
for the application. It's important to note that other GUI options are
available, but upon further analysis, it became clear that Swing and 
JavaFX were the most suitable to our needs.

%---------------------------------------------------------------------

\subsubsection{Swing}
\label{impl:ui:guiframework:swing}

Each member of the group had some limited experience with the Swing 
framework, though not all of it had been positive.
The experience each member of the team had with Swing varied, and
although every member had agreed that their experience had not been
entirely problem-free, we conceded that its integration with the Netbeans
Integrated Development Environment (IDE), discussed in section
\ref{impl:ui:ide:netbeans}, was extremely useful.

However, on investigating the framework more closely it was clear that
Swing was extremely well documented with full API specification
\cite{swingAPI}, and in-depth tutorials \cite{swingTutorial}.
This was a huge part of our decision as we felt that the documentation
provided would be more than adequate to allow us to use the framework
with relative comfort.

%---------------------------------------------------------------------

\subsubsection{JavaFX}
\label{impl:ui:guiframework:javafx}

Another framework considered was JavaFX which no member of the team
had any experience with. 
Some members felt that this was a risk worth taking, given how much
they disliked Swing, discussed above.
In reality JavaFX was only briefly considered and totally disregarded
when, upon brief investigation, JavaFX was still a relatively new
framework, and consequently, comprehensive documentation was not as
readily available for JavaFX as with Swing, particularly when it came 
to troubleshooting on online forums.

In addition, JavaFX required Java 7, which, again, no member of the
group had used before and which was not available, at the time, in the
Level 3 Laboratory where we would be working for the majority of the
year.
It seemed like too much of a risk to try to learn two different
technologies at the same time, while having to provide our own
development platforms, which, with various members of the team never
having used the Linux OS before, could potentially cause a number of
problems.

%---------------------------------------------------------------------

\subsubsection{Decision}
\label{impl:ui:guiframework:decision}

The investigation was carried out by the group's Toolsmith, Ross
Barnie, who presented the evidence discussed in the sections above
regarding the two frameworks to the rest of the team.
With this evidence the team voted in favor of using the Swing
framework with Java 6.

Retrospectively, Swing, and Java 6, are out-of-date technologies and
JavaFX is now packaged with Java 7 \cite{javafxOverview}, so the
application would have been more up-to-date or future-proof had we
used JavaFX.
Additionally, (some of) the computers in the level 3 lab now do have
Java 7 installed upon another project team requesting it, so our fears
over development platform problems were nullified, though this was
only after we had started development.

It was an unfortunate shortcoming of the research into JavaFX that the
group did not know about JavaFX's integration with the Netbeans IDE
which was seen as one of the key differences between the two
frameworks at the time of making the decision.

%---------------------------------------------------------------------
%---------------------------------------------------------------------

\subsection{IDE}
\label{impl:ui:ide}

One concern was that, in some members' experience, using two separate
IDEs was extremely time consuming, particularly while using version
control.
This was mostly due to various metadata that IDEs keep track of in
various files, however this meant that any small change to the source
code would change the metadata and therefore each commit would have to
involve adding it, which would be very time-consuming.

It is because of this experience that the group decided to work from a
single IDE, researched again by Ross Barnie.

%---------------------------------------------------------------------
\subsubsection{Netbeans}
\label{impl:ui:ide:netbeans}

Netbeans is an IDE which the team had had little experience with and had
only used in the context of building applications with GUIs created
using the Swing framework.
There was some trepidation to using Netbeans since most of the team
had associated their problems with Swing with Netbeans itself.
Upon further research, which involved using the IDE to build small
applications, Netbeans started much faster than Eclipse, discussed
below.
And the design interface was very simple and easy to use, with each
element being laid out the way you wish and the associated source code
being generated for you.
This meant that the design layout could be finished very quickly,
rather than spending our time writing hundreds of lines of source code
just for the interface.

In terms of Netbeans' metadata, it was quite minimal and would not
clutter the version control repository to an unacceptable degree.

%---------------------------------------------------------------------
\subsubsection{Eclipse}
\label{impl:ui:ide:eclipse}

The team had substantial knowledge of Eclipse from its mandated use in
Java Programming 2 \cite{javaProgramming2}. Again, our experience of 
Eclipse is somewhat tainted by associations with problems we faced at 
the time, such as a bug on the version for Windows which meant that 
Eclipse would freeze if you tried to copy or paste anything.

In our experience, we found Eclipse to be very slow, both during 
start-up and normal operation. 
Editing-wise, Eclipse was rather cumbersome and not much better than a
text editor.
Also the requirement to bind the ``Workspace'' was seen as a potential
point for confusion and errors.

In addition, the team felt that the missing design interface seen on
Netbeans, discussed above, was a huge disadvantage and would cause a
significant loss of time, simply due to the volume of code we would
have to write instead of being auto-generated.

Members of the team also pointed out that Eclipse has a tendency to
create a large amount of metadata which would clutter the version
control repository.

%---------------------------------------------------------------------
\subsubsection{No IDE}
\label{impl:ui:ide:noide}

It was briefly considered to have no IDE at all and simply use text
editors.
This would allow for extremely fast editing in a very comfortable
environment, since most text editors, such as Vim or Emacs, are highly
customisable and can launch in a matter of seconds.
Text editors would also not require metadata, keeping our version
controlled directories clean.

However, the obvious problem with no IDE is that troubleshooting
problems becomes very tedious very quickly, and unlike IDEs, you
cannot automatically import a missing package or method, nor can there
be any auto-generated code for that matter.

%---------------------------------------------------------------------
\subsubsection{Decision}
\label{impl:ui:ide:decision}

When the evidence above was given to the team, we were also discussing
which GUI Framework to use (as discussed in section
\ref{impl:ui:guiframework}) and it became obvious that integration
with the framework would be key to helping us develop the GUI.

We therefore decided to work with the Netbeans IDE because of the
design interface, minimal metadata, and lack of (known) bugs that
would affect us in any meaningful way.

Retrospectively, this was the correct decision.
Even if we had chosen a different GUI framework, the advantages of the
easy-to-edit design interface far outweigh any problems we had with
it.

%---------------------------------------------------------------------
\subsection{Builds}
\label{impl:ui:builds}

To demonstrate the GUI and the changes we made to it over time, we
will discuss two builds of the system at two crucial points in time.

The first is what the team refer to as the ``demo build'', which was
the first build of the system in general to be used by anyone outwith
the project.
The demonstration itself is discussed in more detail in section
*REFERENCE TO DEMONSTRATION FEEDBACK*.

The second is the current build of the system, which is currently
linked to on the Molecular Methods moodle site to be used by any of
its 160 students.
This build by nature has developed from the demo build in that most of
the changes made were based on the evaluation and feedback we received
from the demonstration itself (discussed in section *SECTION
REFERENCE*)
%---------------------------------------------------------------------
%---------------------------------------------------------------------

\subsubsection{Demo Build}

%---------------------------------------------------------------------
\paragraph{Splash}

Before the demonstration (discussed in section *REFERENCE*), the team
were asked to include an ``overview'' screen to tell the user what
they can expect from the application, as well as show the primer
design rules to remind the user about them.
This can be seen in figure \ref{fig:demoBuild:splash}.

\begin{figure}[h]
  \begin{center}
    \includegraphics[width=0.5\textwidth]{./images/demoBuild/splash.png}
    \caption{
      \label{fig:demoBuild:splash}
      Demo Build, Overview Panel 
    }
  \end{center}
\end{figure}

\paragraph{Sequence Entry}

\begin{figure}[h!]
  \begin{center}
    \includegraphics[width=0.5\textwidth]{./images/demoBuild/sequenceEntry.png}
    \caption{
      \label{fig:demoBuild:sequenceEntry}
      Demo Build, sequence entry panel 
    }
  \end{center}
\end{figure}

Figure \ref{fig:demoBuild:sequenceEntry} shows the next panel,
referred to by the team as the ``sequence entry'' panel.
While based on the design in figure *REFERENCE INITIAL UI DESIGN* it
has been altered slightly to maximise the amount of space to be used
for entering in the sequence, as this is the primary purpose of this
panel.

It was expected of the user to go to the National Center for
Biotechnology Information (NCBI) website and obtain a DNA sequence by
copying it to their clipboard and then pasting this into the sequence
entry panel and this was explained in the accompanying user guide
(appendix *USER GUIDE APPENDIX*).
Although this relied heavily on the users' ability to use keyboard
shortcuts, it was assumed that all students at university level would
at least have an awareness of these shortcuts.
We also assumed that once students were told of these shortcuts, as
they were in the user guide, that they would be comfortable using
them.

%---------------------------------------------------------------------

\paragraph{Area Selection}

\begin{figure}[h]
  \begin{center}
    \includegraphics[width=0.5\textwidth]{./images/demoBuild/areaSelection.png}
    \caption{
      \label{fig:demoBuild:areaSelection}
      Demo Build, Area Selection Panel
    }
  \end{center}
\end{figure}

Following the Sequence Entry panel is the ``Area Selection'' panel,
seen in figure \ref{fig:demoBuild:areaSelection}, which requires the
user to specify the ``target'' sequence, ie the desired output
sequence of the PCR process.
This is accomplished by the user entering the start of the sequence
that they want and the end of the sequence they want, both by the
index of that base in the sequence, into the ``From'' and ``To''
fields at the bottom of the panel.

This would, ideally, be helped by the text pane at the left of the
screen which shows the base number of the first base on its line.
Unfortunately, for an unknown reason, the text panes became misaligned
when viewed from any platform other than the one we were using for
development (the level 3 Computing Science Laboratory computers
running Scientific Linux) and this misalignment can be seen in figure
\ref{fig:demoBuild:areaSelection}.

An addition made to the design discussed in *REFERENCE DESIGN* is the
tabs above the main text area, which allow the user to switch between
the sequence they entered, and its complementary equivalent, generated
by the program.
It was a suggestion by the clients to have this feature as it would
greatly increase the speed at which the user could design the reverse
primer.
Without the complementary tab, not only would the user have to
manually convert the primer to its complementary equivalent, but also
reverse its order, which neither the team or the clients felt was a
useful way for students to spend their time. 

%---------------------------------------------------------------------

\paragraph{Primer Design}

Following the Area Selection panel is the ``Primer Design'' panel,
shown in figure \ref{fig:demoBuild:primerDesign}, which allows the
user to enter forward and reverse primers.

\begin{figure}[h]
  \begin{center}
    \includegraphics[width=0.5\textwidth]{./images/demoBuild/primerDesign.png}
    \caption{
      \label{fig:demoBuild:primerDesign}
      Demo Build, Primer Design Panel
    }
  \end{center}
\end{figure}

One of the design features we had intended to provide was ``dynamic
highlighting'', as referred to by the team, which was going to provide
a highlight around what the user enters into the primer text fields.
This highlighting was unfortunately missing in the demo build due to
time constraints.

However, we had always intended to give feedback to the user should
they break rules of primer design and the demo build version of this
can be seen in figure \ref{fig:demoBuild:primerFeedback}.

Again based on the design *REFERENCE*, this dialogue window shows the
user any rules which they have broken, and which primer the feedback
is referring to.
In terms of user-friendliness and design it could have used a lot of
improvement however this was a feature still in early development.

\begin{figure}[h]
  \begin{center}
    \includegraphics[width=0.5\textwidth]{./images/demoBuild/primerFeedback.png}
    \caption{
      \label{fig:demoBuild:primerFeedback}
      Demo Build, Feedback on User-entered Primer
    }
  \end{center}
\end{figure}

Again, due to time constraints this was not as fully featured as we
had hoped for in the demonstration, however it did display enough
information to give an idea to our clients of what the feedback might
look like in the future (see further discussion in section *REFERENCE
DEMO FEEDBACK SECTION*)

\paragraph{Melting Temperature}

After designing a primer, the user is presented with the ``Melting
Temperature'' panel as seen in figure \ref{fig:demoBuild:meltingTemp}
for the user to evaluate their primers.

In the case of the demonstration, this panel was blocked from the user
unless they had a correct primer.

\begin{figure}[h]
  \begin{center}
    \includegraphics[width=0.5\textwidth]{./images/demoBuild/meltingTemp.png}
    \caption{
      \label{fig:demoBuild:meltingTemp}
      Demo Build, Melting Temperatures of User's Primers.
    }
  \end{center}
\end{figure}

In terms of design, the initial design proved to be near-impossible to
reproduce in Swing and make it look professional and after several
iterations became what it is.
The emphasis on the primers and the melting temperatures in bold means
that the user can easily see their primers and the associated melting
temperatures, while the separator down the center visually separates
the forward from the reverse primer.

Unfortunately the animation was not available in time for the
demonstration and although the button for it was included in this
panel it was disabled.



\section{Models \& Custom Methods}
\label{impl:models}

\subsection{Models}
%- 3 objects, Sequence, Primer, TestResult
%Basic in construction, all classes have toStrings and booleans.

\begin{figure}[h]
  \begin{center}
    \includegraphics[width=0.75\textwidth]{./images/currentBuild/modelClassDiagram.png}
    \caption{
      \label{fig:currentBuild:model}
      Model Class Diagram 
    }
  \end{center}
\end{figure}

The models used in the application are few in number and very simple, BLAH BLAH BLAH BLAH

As seen in Figure \ref{fig:currentBuild:model}, we used three classes to represent the data:
Primer, TestResult and Sequence.

\paragraph{Primer}
The Primer class is purely designed to test user-designed primers. It has 
one attribute, \texttt{code}, the String representing the user's primer
which is tested against in the primer test methods, which make up the 
remainder of the primer. The class is primarily made up of methods 
designed to test \texttt{code} against the various rules described in Section 
\ref{intro:prelims}. The method \texttt{test()} gathers runs all above test 
methods and returns a larger TestResult, indicating if the user's Primer 
is adequate outside of the larger context of the sequence.

\paragraph{TestResult}
TestResult is a class used to format the output of one or multiple primer 
tests. TestResult uses an enumerated type called PassState with values
\texttt{PASS}, \texttt{FAIL} and \texttt{CLOSEFAIL}, the last of which describes 
a state where the primer's value from a test lies outside of the recommended
values, but is close enough to a pass to be acceptable, provided this is 
only the state of a minority of tests. TestResult uses two ArrayLists, one of
PassStates (\texttt{passes}) and another of Strings (\texttt{out}), to keep track 
of the state and informative message to be displayed to the user for each 
test. 

Its methods are concerned with concatenating results into larger 
TestResults. \texttt{perfect()} will return true if all entries in 
\texttt{passes} equal a \texttt{PASS}. \texttt{adequate()}, the method that is 
checked to gate the user's access past the Primer Selection panel returns 
false if any of the tests returned 	exttt{FAIL}, or if more than 60\% 
returned \texttt{CLOSEFAIL}, and returns true otherwise.

\paragraph{Sequence}
This class contains two Strings, \texttt{oStrand} and \texttt{cStrand}, 
representing the strand of DNA that the user took in and the ''complementary''
strand that is generated when the sequence is constructed. Integers 
\texttt{start} and \texttt{end} represent the indexes of the start and end of 
the selected area in the sequence, and Primers \texttt{fPrimer} and 
\texttt{rPrimer} are, obviously, representations of the user's primers.
 
\texttt{parser()} is used for both sequence entry and primer input, by taking in
a String and returning a new String with all non-\texttt{atgc} characters removed.
\texttt{complement()} is a very simple function used throughout the application
that takes in one character representing a base, and returns its complement, i.e.
\texttt{complement('a')} would return \texttt{t}. \texttt{isUnique()} and \texttt{
tempDifference()} check the user's primers against the rules concerned with the
larger Sequence. \texttt{primerTest()} uses all other test methods to return a
TestResult in accordance with the  																										% phrasing?


\subsection{Primer Checking}
The 'Primer Checks' methods are implementations of the established rules
and guidelines which are used in the process of Primer Design (seen in
the preliminaries section) to evaluate the effectiveness of a given 
Primer when used in the PCR process.

As with many other aspects of the project, the primer checks were split
between the two members of the 'back-end' sub-team. As well as making
this task more manageable, this approach offered the added benefit of
limiting the researching of primer design rules to 2 members, who each
only had to learn how to apply about 4 methods each.

The complexity of these rules varied greatly in difficulty, from
trivial checks such as “Primers should end in a base 'g' or 'c'” to
challenging checks such as checking how likely it is that one end of a
primer  will anneal to the other. The majority of these methods fell
into the first category and were fairly straightforward to implement,
however,  implementing the more difficult methods posed a serious
challenge.

Firstly, none of the members of the team had any experience with PCR or
Primer Design prior to the start of the project so every rule had to be
thoroughly studied and understood before beginning to design the
methods. However, even after spending time learning how the design
rules and guidelines are used, some methods still proved problematic.

\subsubsection{Melting Temperature}
The melting temperature check was among the easiest to implement, as we
were given a standard formula for calculating the temperature at which a 
primer would melt (\texttt{2*sum(a, t) + 4*sum(a, t)}) and the desired 
range in which a primer should rest (between roughly 50\degree C and 
60\degree C).

\subsubsection{Uniqueness}
To check that the primer is correctly placed in the system, it is first checked
that the primer cannot be found in the wrong strand (ie. that the forward
primer cannot be found on the complementary strand). When this is confirmed,
it is then checked that the correct strand contains one and only one instance
of the primer, and then that this instance is positioned correctly, relative
to the area of the sequence to be replicated.

\subsubsection{Primer Content}
Many of the primer rules were trivial to implement in the system, given
the string representation of a primer. Such rules include ensuring the
length of the primer lies between 20 and 30 bases, checking the last base
of a primer and confirming that no base is repeated more than 4 times
in a row. 

The melting temperature was among the easiest to implement, having been
given a standard formula for calculating the temperature at which a primer
would melt (2*sum(\verb£a, t£) + 4*sum\verb£a, t£) and the desired range
in which a primer should rest.




\subsection{Dynamic Primer Highlighting}
\label{impl:models:dynHigh}

Dynamic Primer Highlighting was suggested early on in the requirements
elicitation process by the clients as something that would be very
helpful to students studying primer design. The initial specification
for this feature was as follows:

\begin{itemize}	
\item“As the user enters their choice of primer in one of the boxes at the
top of the page, instances of the primer should be highlighted in
real-time in the corresponding box below containing the DNA sequence for
the strand on which this primer should appear.”
\end{itemize}

In a later client meeting the following addition was made to improve the
effectiveness of the feature:

\begin{itemize}
\item“Primers should be highlighted in different colours to indicate their
suitability.”
\end{itemize}

This was a feature that all members of the team appreciated from a very early
stage because, even with our limited knowledge about primer design we
could see how this form of immediate feedback had the potential to
improve the usability of the system if it were to be implemented well.
Due to this level of popularity among both the clients and the team
members the feature was given high priority. However, since we knew it
would be complex to implement and would require calls to other planned
modules of the system, such as the primer checking functions, it was
also decided that this should be one of the last features to be
implement.

This feature turned out to be one of the most problematic features to
implement as it involved using features of Java which we had no real
experience with, specifically the Swing classes \texttt{ChangeListener},
\texttt{Highlighter} and \texttt{Painter} as well as integrating other
modules of our code to provide the required primer checking. This level
of difficulty meant that this feature actually took multiple attempts to
implement correctly.

\subsubsection{Implementation}

The implementation of this feature can be split into two main components:

\begin{itemize}
\item The code to listen for user input in the primer entry text fields
and call the appropriate methods to deal with the input.
\item The runnable objects \texttt{searchO} and \texttt{searchC} which
are called by the listener and update the highlighted text.
\end{itemize}

\subsubsection{Listening for Input}

The first challenge we were met with when implementing this feature was
deciding how to listen for user input in the
\texttt{forwardPrimerTextField} and \texttt{reversePrimerTextField}.

Due to the fact that only one strand's display tab would be shown at any
given time, only the TextField related to the active tab would have to
be listened to and any user input in the other field could be ignored
until the active tab changed. This meant that only one listener would be
needed and the TextField it was listening for updates on could be
switched when required.

Since this feature was implemented late on in the development cycle,
code already existed to deal with changes upon switching tabs.
Therefore, we were able to add the calls to switch the TextField being
listened to into the existing \texttt{updateLineNums()} function which
was only being used to update the line numbers upon switching between
single and double-stranded views.

As well as switching the target of the listener upon switching tabs, a
a call must be made to the search function related to the new tab
since any user input into the corresponding TextField while the tab was
not active will not yet be reflected in the display tab. This call is
made using and \texttt{invokeLater()} call of the runnable search
function as oppposed to a standard \texttt{invoke()} call to allow the
system to continue listening for updates if a call to one of the search
functions takes a long time to process.

Once we had worked out how we were going to switch the listener focus
all that was left to do in terms of listening for input was to add code
to the \texttt{DocumentListener} functions \texttt{insertUpdate()} and
\texttt{removeUpdate()} to run the appropriate search function. All this
required was an if statement to determine the source of the update and
inside the if statement a call to the appropriate search.

\subsubsection{Runnable Search Functions}

After dealing with listening for user input, the next task was to create
the functions to search for the user's primer in the DNA sequence and
highlight the appropriate sections.

To perform this task we created two similar methods: \texttt{searchO()}
to search the parsed original strand, \texttt{parsedO}, and update forward
primer highlights and \texttt{searchC()} to search the parsed
complementary strand, \texttt{parsedC}, and update reverse primer
highlights.

The first task these functions perform is to remove all existing
highlights from the display. The other option here was to keep a list of
all highlights and their positions in the sequence then check each one
individually to determine if they were still valid after the latest
input and highlight an extra base at the start of end of the old
highlight. This list based solution was not chosen because, for example,
if a user's first input was a single letter then the program would
potentially have to store the positions of thousands of bases and
perform thousands of comparisons on their next input.

The next step is to get the position of the first instance of the primer
in the sequence using the java \texttt{indexOf()} function. This is then
used as part of the condition of the while loop which carries out most
of the functionality in these methods. The while loop is excecuted while
the result of the \texttt{indexOf()} call is not negative (i.e. the
primer is found in the text being searched) and the user input has a
length greater than 0 (i.e. The call to the function was not because the
user deleted everything in the text field).

Inside the while loop the colour to highlight the current instance is
determined by the following if statement from \texttt{searchC}:

\begin{lstlisting}
    if (sC.length() > 15){
        Primer rPrimer = new Primer(Primer.reverse(sC));
        rTest = new model.TestResult();
        rTest.addFull(rPrimer.test());
        rTest.add(
            rPrimer.isUnique(PrimerDesign.start.getInSequence(),
                'c'));
        if (rTest.perfect()){
            activePaint = perfectPaint;
        } else if (rTest.acceptable()){
            activePaint = acceptPaint;
        } else {
            activePaint = failPaint;
        }
    } else {
        activePaint = failPaint;
    } 
\end{lstlisting}

Here, \texttt{sC} is the user's primer input and \texttt{perfectPaint},
\texttt{acceptPaint} and \texttt{failPaint} are different coloured
\texttt{Painter} objects initialised when \texttt{PrimerSelectionPanel}
is first loaded. The only difference between the three \texttt{Painter}
objects is their colour with \texttt{perfectPaint} being blue,
\texttt{acceptPaint} being yellow and \texttt{failPaint} being red.
These were initially green, yellow and red but green was eventually
substituted for cyan due to concerns about colour-blind users.

The late stage at which this feature was implemented meant that the
process of evaluating the user's primer was a trivial task as all the
methods check the primer were already written, as were the methods to
evaluate the success of the primer overall. So the method simply has to
create a new \texttt{Primer} instance with the user's input and create
an instance of \texttt{TestResult} to store the results of the primer
checking methods. Then the method calls the \texttt{perfect()} and
\texttt{acceptable()} methods to determine the colour the primer should
be highlighted.

Once the colour of highlight has been determined the highlight is
applied to the correct section of the DNA sequence using the following
code:

\begin{lstlisting}
    endC = indexC + sC.length();
    highC.addHighlight(realIndex(indexC + checked, 10),
        realIndex(endC + checked, 10), activePaint);
\end{lstlisting}

\texttt{endC} is calculated by adding the length of the user's primer to
the index of the current instance of the primer in the sequence, 
\texttt{indexC}. \texttt{realIndex} is a function which converts the
indices in the parsed sequence into the indices needed for displaying
the highlights in the display panes.

Once the current instance of the primer has been highlighted, a 
\texttt{substring()} is performed on the relevant parsed sequence to
remove everything up to the end of the current primer instance. The
index of the first instance of the primer in this new, shorter sequence
is then found and if this index is greater than or equal to 0 then the
while executes again to highlight this new instance of the primer.

After all instances have been highlighted the parsed sequence is reset
to the state it was in before the function was called.













\section{Animation}
\label{impl:anim}
\paragraph{Requiredfunctionality}
Our initial plans for the animation were to make it depend on the actual user input in the exercise itself, particularly taking the sequence and the primers from the previous stages of the application. This was mainly to differentiate more from other similar PCR animations, like the ones demonstrated to us by the biology staff (\cite{anim1} , \cite{anim2}). Because of the high customisation requirements set by the task at hand and to provide maximum compatibility throughout the project we decded to develop the animation on Java utilising Java graphics package and Swing together, with the animation logic being developed from scratch. However, towards the end of the animation development it became apparent that the sequence required to be presented in the animation is just too large to be reasonably scaled with the individual nodes being visible. To extrapolate, the individual nodes were hard to make out as soon as the target PCR area reached about 150 nodes in length, which was not enough as the area could be well over that limit. Finally, as the clients were satisfied with a static animation, it was decided that we would just use the animation with a sample input that  was adequately sized. It was also suggested by the biology staff that the individual nodes' color coding doesn't need to be explained in the animation as it was clear enough that the different colors represent different node types. Otherwise the aforementioned shift in the overarching animation design didn't affect it in any way. The final animation sreenshots and explanation of its individual parts are provided below.

\paragraph{Implementation}
The whole animation is controlled with a large set of conditional statements by a swing timer that constantly recalculates the time passed from the start of the animation, pausing if it reaches the end of a stage until a button press changes the stange and therefore sets the time passed to a particular value. The models used in the animation were developed in the Inkscape vector graphics editor and are the different type of the individual nodes models, the taq polymerase model and three models for three different states of the thermometer. The individual nodes are then drawn to make a sequence at a particular location.

\paragraph{Functionalityexplained}
The animation is split into 7 stages, with each one having a text about the animation at the bottom. The bottom line in small font size indicates the current stage of the animation, with the complicated stages having a short explanation of the stages in brakets, and the state of that stage, which can be "in progress" or "finished". The thermometer in the bottom right part of the sreen is shifting between three possible states, according to the three temperature levels required by PCR: 55°C, the average melting temperature of primers; 72°C, the temperature at which the Taq DNA polymerase synthesizes complementary strands; 95°C, the temperature of denaturation (DNA individual strand separation). The four buttons above the text are: "Close", to close the application completely; "Restart", to go to the start of the exercise; "Previous" and "Next" to navigate between the animation stages. The animation won't proceed until the "Next" button is pressed, something which was suggested by the biology staff, as they pointed out that the user might not be able to read the text in the time provided. Finally, the PCR animation itself is in the top part of the screen.

\paragraph{AnimationScreens}

In the stages 0 and 1, the user is explained what PCR is and what ingredients it requires as the temperature goes up to 72°C.

\begin{figure}[h]
  \begin{center}
	\includegraphics[width=0.6\textwidth]{./images/AnimImpl/Stage1}
    \caption{
      \label{fig:AnimImpl:stage1}
      Animation, stage1
    }
  \end{center}
\end{figure}


In stage 2, Melting and Annealing, the strands are separated and the primers bind to them, with the temperature level varying accordingly.

\begin{figure}[h]
  \begin{center}
	\includegraphics[width=0.6\textwidth]{./images/AnimImpl/Stage2}
    \caption{
      \label{fig:AnimImpl:stage2}
      Animation, stage2
    }
  \end{center}
\end{figure}

In stage 3, Adding nucleotides, the taq polymerase creates a complementary copy of each strand, with the temperature once again raised to 72°C.

\begin{figure}[h]
  \begin{center}
	\includegraphics[width=0.6\textwidth]{./images/AnimImpl/Stage3}
    \caption{
      \label{fig:AnimImpl:stage3}
      Animation, stage3
    }
  \end{center}
\end{figure}

In stages 4 and 5 another cycle of PCR is shown and the required sequence is generated for the first time.

\begin{figure}[h]
  \begin{center}
	\includegraphics[width=0.6\textwidth]{./images/AnimImpl/Stage3}
    \caption{
      \label{fig:AnimImpl:stage5}
      Animation, stage5
    }
  \end{center}
\end{figure}

Finally, the last two stages explain how many copies of the target sequence is produced in subsequent cycles and the user is explained that he completed the exercise and thanked for participation.

\begin{figure}[h]
  \begin{center}
	\includegraphics[width=0.6\textwidth]{./images/AnimImpl/Stage3}
    \caption{
      \label{fig:AnimImpl:stage6}
      Animation, stage6
    }
  \end{center}
\end{figure}

%=====================================================================
\chapter{Evaluation}
\label{eval}

\section{Testing}
\label{eval:testing}
% ADD TO THESE COMMENTS AS YOU SEE FIT 
%
% Testing - Internal Testing, give to the supervisors for feedback, changed according to their requirements
%
% Demo - dry run through with students. Given user guide and feedback sheets, changed according to their feedback
%
% Questionaire - Sent out with jar file on moodle, feedback from students on 'finished' application

As with any software development, testing is a nessecary component. During our time with the PCR application, testing begun
almost as soon as implementation started. Our testing fell into two categories: \\

\begin{itemize}
\item Testing of the backend, such as primer checking formulas etc.
\item Testing of the user interface
\end{itemize}






\section{Demonstration}
\label{eval:demo}
On the 8th February, our team carried out a demonstration run of our software with a group of users connected to the School of Life Sciences. In this test, we provided users with a comprehensive user guide and asked them to carry out a set of tasks using, what we called at the time, the demo build of the application. With this version of the program we had decided to disable the animation, as the animation had been developed using Java 7 \cite{Java7SwingAPI}, whereas the rest of the application had been developed using netbeans and Java 6, as had been decided before implementation, (See: chapter \ref{chap:impl}). However, the rest of the application we believed to be functioning at the time of the demo run, minus some major features we had yet to implement, such as the dynamic highlighting feature implemented in the final version (See: Section \ref{impl:models:dynHigh}).

\subsection{Running the demonstration}

The demonstration took place between 12pm and 1pm in a computer lab in the Wolfson building, with a group of about 12 participants, consisting of students, demonstrators, lecturers, and the clients. The computers on which the application was run were all identical, all running on Windows 2000. We issued each participant with a user guide (See related documentation), detailing how to load the application from moodle and get it running, and then gave them clear instructions as to how to use the application to test for correct primers. Three members of our team (Ross Eric Barnie, Murray Ross and Ross Taylor) were on hand to help participants with any problems they may have had with the program. We let them work with the application for about half an hour before asking them to finish their work and sit down to have a round table discussion about their experiences with the system, and to fill out an evaluation sheet which we also provided them with.

\subsection{Feedback from demonstration}

At the round table discussion, we used a sound capturing device to record what people had to say about the application, which was later annotated in a text document (See: Appendix \ref{app:roundtableFeedback}). Also, we asked participants to fill out an evaluation feedback sheet at the end which was then read by our team and the points were collated and annotated into another text document (See: Appendix \ref{app:demonstrationFeedback}). The main points that were taken from the feedback provided by the demonstration were:

\begin{itemize}

\item That the primer rules were unbreakable - you could not progress to the melting temperature screen without passing all the rules set by the program. In a typical DNA sequence, finding a 'perfect' primer is extremely hard, and the rules should be there more as guidance, not as set in stone.
\item Further to the first point, using the words 'PASS/FAIL' is too harsh, and the capitalisation should be taken out.
\item That the interface shown to the participants that whilst clear, was very clinical, very '90s'.
\item As a teaching tool, the application was definitely better than any paper version previously provided by the School of Life Sciences.
\item The purpose of the application was misconstrued - it isn't there to find primers for you, it is there for the user to find them and test them.
\item It would be helpful if the program could check each primer individually, and not together at the same time.
\item The application required the user to use keyboard shortcuts to copy/paste, and that a few participants did not use these shortcuts in common use and preferred graphical buttons to do these functions.
\end{itemize}

We then collected in the evaluation feedback sheets from the participants. The main points that we took from these sheets were:

\begin{itemize}

\item In general, it was always clear how to progress in the application from stage to the next.
\item Aesthetically, the application did not have a ostentatious design but it met the demands and expectations of the user.
\item The rules provided in the \emph{Primer Design Rules box} were very helpful, but should point out these are only guidelines and not set in stone.
\item If you infringed on a rule, it should allow you to pass through anyway, perhaps using an \emph{override} button.
\item Some users found it to improve their knowledge of primer design.
\item It was noted that if you had a higher melting temperature in your first primer (i.e. 64) than the second (i.e. 62), but both were still in the melting temperature range, it would tell you that they were not within the required temperature range of each other, yet they clearly were.
\item Reverse primers would dissapear from screen, forcing users to write them down if they wanted to remember them.
\item A helpful idea would be that as the user entered their primer it highlighted it on the sequence for them to see.

\end{itemize}

With this feedback, we were able to decide on which changes were required to the program.

\subsection{Changes proposed after feedback from demonstration} 

After we had met to discuss the feedback, the following changes to the application were decided upon:

\begin{itemize}

\item Implementation of the dynamic highlighting feature (See: Section \ref{impl:models:dynHigh}) - whilst this was initially our intention during development, we had failed to implement it in time for the demo. However, from the feedback above where users were losing their primers during usage of the application, and also for general accessibility purposes, we decided implementing this feature had become a priority.

\item Reduce the strictness of the primer design rules. From the feedback we received above, we found that users had previously used these rules as guidelines to follow in primer design, and that they were not necessarily adhered to all the time. Therefore, we decided that instead of requiring a 100\% success rate in order to reach the melting temperature screen, we would allow users to 'override' the rules after 3 attempts, making the rules more like the guidelines users were expecting as opposed to set in stone.

\item After users had notified us of the bug which tells a user that their temperatures were not in range when they were, we decided to analyse the formula for calculating the temperature in order to fix this fault.

\item Some users had noted in the roundtable discussion that they would like to be able to check each primer individually, as opposed to both at the same time. To accommodate this, we decided to implement individual primer checker buttons on the primer selection panel to allow users to check each primer separately against the guidelines.

\item Certain users were confused by the lack of copy/paste buttons on the user interface, and did not know the keyboard shortcuts to use in order to get around this. To allow users more accessibility, we decided to implement graphical buttons for copy/paste.

\item The feedback we received indicated that they though the program was too harsh and clinical when it came to notifying users of their test results for their chosen primers. To help this, we decided to replace capital letters in the pass/fail messages to reduce harshness.

\end{itemize}


\section{Questionnaire}
\label{eval:question}
In order to test the final system against the requirements set out in
section \ref{design:reqs}, the team decided to produce a
questionnaire which would be given to students using the application
for them to give us feedback.
The raw data for this feedback is shown in appendix
\ref{app:questionnaireResponses}.

While other methods of testing exist (e.g. experimental) it was felt
that with the time constraints in place we would not be able to run a
test requiring our direct involvement and that we would receive more
data from a questionnaire.

%---------------------------------------------------------------------
\subsubsection{Questionnaire System}

Many questionnaire services exist on the internet, such as
SurveyMonkey \cite{surveyMonkey}, FreeOnlineSurveys
\cite{freeOnlineSurveys}, and Google Docs Forms
\cite{googleDocsForms}.

The team decided to use the Google Form, simply because most members
of the team have a Google account, and because it was free.

On closer inspection it could also export its responses to a Google
Docs Spreadsheet which itself could be exported to a variety of
formats.

%---------------------------------------------------------------------
\subsubsection{The Questions}

Generally, the questionnaire needed to be as concise as possible to
avoid any data being corrupted by frustration at the questionnaire.
To this end we kept the number of questions in general to a minimum
and only asked for a description if it was necessary.

\paragraph{Skill Level}
At the demonstration, the team realised that not every user will be in
the expected age bracket of 18 to 20-years-old who have used computers
their entire lives.
This led to the decision that the feedback should take the user's age
into account, to ensure that there are no patterns in the data
suggesting a particular age group struggles with the application.

In addition, we ask the user for their ``Confidence'' with computers
in general, on a scale from 0 to 5 ie no middle option so there has
to be a bias to confident or not confident.
This completely subjective question should allow us to see if people
who perhaps do not use computers on a daily basis handle the
application, since we believe we made the application (and
accompanying user guide in appendix \ref{app:userGuideCurrent}) as
user-friendly as possible.

We are also aware of some colour-use in the program that could be
problamatic for colour-blind people so we ask the user if they are or
not.

To get an idea of the user's knowledge of primer design, we ask the
user for their self-assessed understanding before using the system, on
a 0 to 5 scale.

All of this data provides us with the baseline skill level with
computers and with primer design of the user, before they use the
system.

\paragraph{User's Experience with the System}
The following questions were designed to provide us with feedback on
the user's experience with the system.

The first of these follows the self-assessed understanding of primer
design before using the system, with a self-assessed understanding
after using the system, on the same scale.
This will provide us with test data towards the requirement to teach
students about primer design and PCR.

Ideally, as discussed in section \ref{design:reqs}, the system would
be used as a revision tool in students' own time and/or in a
laboratory setting with tutors on hand to help the student with any
problems.
To see if we met this requirement, we asked the user if they would use
the system to study one of the following options:
\begin{itemize}
\item Both Primer Design and PCR
\item Just Primer Design
\item Just PCR
\item Neither
\end{itemize}
Which provides the user, and the team, with every possible answer to
the question, in the most concise way possible.

Following this, we decided to ask the user if they felt they were
given enough information from the application.
In retrospect, this question should have been clearer in that it
should have specified to disregard the user guide.
We wanted to ask this to assess how easily people would be able to use
the application without the user guide.

Technical difficulties were fairly common in the demo build (section
\ref{eval:demo}) but since members of the team were present, we could
instantly know about them.
However since the system was now being used from where the user
happened to be, we had no direct way to be told of any technical
issues or bugs.
To this end, we asked the user for any technical issues they found
while using the application.

Lastly, we ask for any further comments, in case the user wanted to
tell us something we had not asked before this point in the
questionnaire.
To keep it light-hearted, we recommended telling us a joke in the
description of the question, needless to say we had some interesting
feedback for this particular question.

%---------------------------------------------------------------------
\subsection{Feedback Analysis}

In total we received 15 responses, see appendix
\ref{app:questionnaireResponses} for the raw data.
These responses have yielded some interesting results with a few
anomalies.

\subsubsection{Learning}

Of course, the main objective of the system is to help users revise or
solidify the idea of primer design within PCR, if we have not managed
this, we have failed our users.

\begin{figure}[h]
  \begin{center}
    \includegraphics[width=\textwidth]{./images/perceivedLearning.png}
    \caption{Perceived Learning Chart}
    \label{fig:feedbackAnalysis:perceivedLearning}
  \end{center}
\end{figure}

If the application was perfect in this regard and a graph was plotted
for response against score given to the ``Before'' and ``After''
understanding questions, the graph would show that the ``After''
responses line was consistently above the ``Before'' line, implying
that users always see an improvement in understanding.
Unfortunately, as can be seen in figure
\ref{fig:feedbackAnalysis:perceivedLearning} which used the data in
appendix \ref{app:questionnaireResponses}, this is currently not the
case.

One third of responses stated that their understanding of primer
design stayed the same before and after using the system and, more
concerningly, one response claimed that using the system reduced their
understanding.
This still means that a majority of users, 60\% of users who responded
to the questionnaire, found using the system increased their
understanding, however this still implies that 40\% of users did not
see an improvement in understanding.

On closer inspection of the users who saw no improvement and stated
that they would not use the system for studying primer design or PCR,
only one gave additional information as to their problem.

This responder stated that they were not confident in how to design
primers and that they expected the system to guide them through the
process.
They go on to say that they had no way of getting additional
information or hints as to how to progress and that there should have
been some way of retrieving this information.
On noting this, the team agreed that, while the ``Primer Design
Rules'' and ``Primer Feedback'' functions of the system were useful to
someone who had a basic understanding of primers, newer users would
struggle since the rules and feedback themselves are not explicit
in telling the user how they can improve their primers, though it is
very heavily implied.
The team agreed that while this was an isolated case, it may have been
the reason for the other responses which stated no improvement in
understanding, and has been added to future work (see chapter
\ref{future}).

\subsubsection{Shortfalls of the Questionnaire}
We received no data to insinuate that a particular age group had
problems with the application, with only 3 responders stating that
their age was higher than 20, each with varying base skill level and
responses.
Therefore we do not have enough data to suggest an age-related
correlation.

None of the reponding users stated that they were colour blind, so we
are unable to state that colour blind people would be able to use the
system as easily as those who are not.

The question regarding information from the application was evidently
unclear to the students who frequently responded by talking about the
user guide, rather than the application.
This should have been clarified and perhaps a separate question was
needed for the user guide itself.

\section{Future Work}
\label{eval:future}
We don't plan to develop the project further unless asked and given specific suggestions by the Biology Department. We believe the application is sufficiently presentable and working adequately. If, however, an important imlementation or design flaw is surfaced in subsequent student testing, we will do our best to resolve the issue. One obvious part that could be improved is the animation, as it can now be made in Flash or other platform more suitable for static animations, whhich would make the animation more aesthetically advanced. However, it would require a lot of time resources used for the new platform familiarisation as well as the development itself, so we decided it is not worth it right now. 

%=====================================================================

\chapter{Conclusion}
\label{conc}

%=====================================================================
\appendix

\chapter{Glossary}

\begin{description}
\item[PSD or PSD3]{Professional Software Development 3, a compulsory
    level 3 Computing Science Course designed to teach students how to
    develop software in a professional manner}
\item[GitHub] {GitHub refers to the website \url{github.com} which
    hosts the git repository for our project's documentation and
    implentation at \url{https://github.com/Dan-McElroy/Team-Project--Q}}
\item[Git] {Git is a version control system used by Team Q to keep
    track of all digital files related to the system. Not to be
    confused with the term of `endearment' in the English language.}
\item[TP or TP3] {Team Project 3, a compulsory Level 3 Computing
    Science module where students are required to produce some piece
    of software. In our case, this system.}
\item[Portability] {A system which is portable is able to be used on a
    different computer to the one on which it was developed with no,
    or minimal, changes.}

\item[Github] An online project management tool
\item[PSD3] Professional Software Development 3
\item[Trac] An online project management tool
\item[bash] A Unix command shell written for Gnu Project
\item[pdf] A file format used to represent documents
\item[LaTeX] A document markup language used for all project documentation
\end{description}
\end{description}


\chapter{Questionnaire Responses}
\label{app:questionnaireResponses}
\includepdf[pages=-]{questionnaireResponses.pdf}

\chapter{Roundtable Feedback}
\label{app:roundtableFeedback}
\includepdf[pages=-]{roundtable.pdf}

\chapter{Demonstration Feedback}
\label{app:demonstrationFeedback}
\includepdf[pages=-]{feedback_eval.pdf}

\clearpage
\bibliographystyle{plain}
\bibliography{references}


\end{document}
