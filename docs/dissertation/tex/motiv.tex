At the beginning of the project we were sent three links to systems
currently in place which attempt to make learning this process more
interactive and/or visual.

The first was a video hosted on YouTube \citep{youtube:taqExtension},
made by demonstrators within the School of Life Sciences.
During its eighteen second duration, the video shows various elements
of the PCR process including change in temperature and the role of the
primer.
However, it was commented by the team and by the clients that it was
insubstantial in terms of information delivery, several of the stages
of PCR are omitted with no mention of primer design, and in terms
of interactivity.

Another video hosted on YouTube \citep{youtube:PCR}, currently referred
to on School of Life Sciences' website, is similar in style to a
lecture with slides and a voice-over which repeats the textual
information on each slide.
While this video is far more informative than the previous one, with
each stage of PCR clearly described, and with visually pleasing
animations, it lacks in explicit primer design and again in
interactivity.

Videos and multimedia in general have been questioned as teaching
aids.
Simply because the information is in video or multimedia format
does not necessarily mean that it is benefitting the learning of its
viewers, or creating the correct environment to encourage learning.
Interactivity, along with other factors, are key to engaging people to
learn \citep{gamingRedefines2004}.

Finally, an animation from the University of Utah, titled ``PCR
Virtual Lab'' \citep{genScienceCenter2012}.
This is a much more interactive experience and allows the user to use
virtual pipettes in order to simulate what you would do in a lab
situation when performing PCR.
Additionally, the information it provides, while slightly basic in the
beginning for our target users, is extensive and very informative to
the novice user, such as Biology-illiterate Computing Scientists.
While this is a much more interactive and, compared to the
alternatives described above, much more informative experience, it
fails to provide the user with the theoretical background information,
particularly on primer design, required to fully understand the
process and why the reaction occurs.
