\frame{\titlepage}

\begin{frame}
\frametitle{Requirements gathering - Initial data}
At the first meeting with the clients we were provided a document outlining the requirements for the end product. The aim of the project was described as follows:
\begin{quotation}
To design a PCR-primer design exercise to complement teaching of a
Molecular Methods course to Level 3 Life Sciences Undergraduates. This
exercise will be integrated into a new part of the lab which we are
designing based around diagnosis of HIV using PCR. You will need to
understand the theory behind PCR and primer design in order to achieve
this.
\end{quotation}
\end{frame}    

\begin{frame}
\frametitle{Requirements gathering - System Scope}
\begin{itemize}
\item{The application should be usable in a teaching environment or by people on their home computers.}
\item{It should function as an interactive, step-by-step guide through 
the process of PCR on a DNA strand of the user's choice.}
\item{The system should also provide the user help with completing each 
task by providing relevant rules for each task.}
\item{When the user has completed the required tasks, the system will then show an animation of the PCR reaction taking place.}
\end{itemize}
\end{frame}

\begin{frame}
\frametitle{Requirements gathering - feedback}
Throughout the project, we maintained weekly meetings with our supervisor and the clients, and this allowed us to iteratively improve our design and, later on, the application itself. We received substantial feedback on the User Interface and the primer check handling, which allowed us to make the application more valuable as a teaching tool.
\end{frame}

