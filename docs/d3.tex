%%%%%%%%%%%%%%%%%%%%%%%%%%%%%%%%%%%%%%%%%%%%%%%%%%%%%%%%%%%%%%%%%%%%%%%%%%%%%%

\documentclass{l3deliverable}

%%%%%%%%%%%%%%%%%%%%%%%%%%%%%%%%%%%%%%%%%%%%%%%%%%%%%%%%%%%%%%%%%%%%%%%%%%%%%%

\usepackage{graphicx}%
%
%\usepackage{svn-multi}%
%\svnid{$Id: d3.tex 2500 2011-09-21 15:56:43Z tws $}

\version{1.0 Made \today by Ross Eric Barnie}

\usepackage{tabularx}%
\usepackage{url}%
\usepackage{usecasedescription}%

%%%%%%%%%%%%%%%%%%%%%%%%%%%%%%%%%%%%%%%%%%%%%%%%%%%%%%%%%%%%%%%%%%%%%%%%%%%%%%
%% Check these macro values for appropriateness for your own document.

\title{Requirements Document}

\author{Ross Eric Barnie \\
        Dmitrijs Jonins \\
        Daniel McElroy \\
        Murray Ross \\
        Ross Taylor
      }

\date{\today}

\deliverableID{D3}
\project{Team Project 3: Interactive Primer Design}
\team{Q}

%%%%%%%%%%%%%%%%%%%%%%%%%%%%%%%%%%%%%%%%%%%%%%%%%%%%%%%%%%%%%%%%%%%%%%%%%%%%%%

\begin{document}

%%%%%%%%%%%%%%%%%%%%%%%%%%%%%%%%%%%%%%%%%%%%%%%%%%%%%%%%%%%%%%%%%%%%%%%%%%%%%%

\maketitle

\tableofcontents

\newpage

%%%%%%%%%%%%%%%%%%%%%%%%%%%%%%%%%%%%%%%%%%%%%%%%%%%%%%%%%%%%%%%%%%%%%%%%%%%%%%
%% Standard section for all documents

\section{Introduction}

\subsection{Identification}

This document outlines the Requirements Specification of the system to
be implemented, to be approved by the clients.

\subsection{Related Documentation}

% NEED RELATED DOCUMENTATION

Initial Problem Specification:
\begin{itemize}
\item{See Section \ref{sec:clientMeetingDocs}} %Attach this if possible
\end{itemize}

PCR Design Principles (Summary of Meeting on 10/10/2012)
\begin{itemize}
% This needs to be produced: Dan given the task
\item{See Section \ref{sec:clientMeetingDocs}} 
\end{itemize}

List of Animations Involving PCR Provided by Clients
\begin{itemize}
\item{\url{http://www.youtube.com/watch?v=XXkG6m3yT1M&feature=youtu.be}}
\item{\url{http://ibls.moodle.gla.ac.uk/mod/resource/view.php?inpopup=true&id=33097}}
\item{\url{http://learn.genetics.utah.edu/content/labs/pcr/}}
\end{itemize}

There are various other documents including meeting minutes which are yet to
be collated.

\subsection{Purpose and Description of Document}

This document will serve as an agreement between the client and Team
Q as to the requirements of the system. Any disagreement between the
two parties regarding these requirements will be documented here along
with details on how this was resolved.

\subsection{Document Status and Schedule}

Currently this document is in draft and scheduled to be finalised
pending review and approval from the clients on 7th November 2012.

After which the document will be updated iteratively as changes
arise and will be re-approved by the clients before being re-released.

\section{Extended Problem Defintion}

%%%%%%%%%%%%%%%%%%%%%%%%%%%%%%%%%%%%%%%%%%%%%%%%%%%%%%%%%%%%%%%%%%%%%%%%%%%%%%

Give an extended description of the problem here.

%%%%%%%%%%%%%%%%%%%%%%%%%%%%%%%%%%%%%%%%%%%%%%%%%%%%%%%%%%%%%%%%%%%%%%%%%%%%%%

\section{System Scope} %TODO

Give an overview of the system here, in the context of the surrounding
environment.  Use case diagrams can be used to illustrate the
interactions between actors in the environment and the system.

You should explain the assumptions you have made in defining the
boundary of the system (i.e. what the system will and will not do).

Describe any conflicts in requirements expressed by different
stakeholders, how you resolved them and why.

%%%%%%%%%%%%%%%%%%%%%%%%%%%%%%%%%%%%%%%%%%%%%%%%%%%%%%%%%%%%%%%%%%%%%%%%%%%%%%

\subsection{System Actors} 

Give descriptions of each of the actors that you have identified as
interacting with the system.

%%%%%%%%%%%%%%%%%%%%%%%%%%%%%%%%%%%%%%%%%%%%%%%%%%%%%%%%%%%%%%%%%%%%%%%%%%%%%%

\subsection{Domain Model}

Explain the elements of the domain here.

%%%%%%%%%%%%%%%%%%%%%%%%%%%%%%%%%%%%%%%%%%%%%%%%%%%%%%%%%%%%%%%%%%%%%%%%%%%%%%

\section{Use Case Descriptions}

This is a collection of use case descriptions (one per use case).
Think carefully about how to group these descriptions in the document.
You can use the template style provided to format your descriptions:

\begin{UseCaseTemplate}
\UseCaseLabel{}
\UseCaseDescription{}
\UseCaseRationale{}
\UseCasePriority{}
\UseCaseStatus{}
\UseCaseActors{}
\UseCaseExtensions{}
\UseCaseIncludes{}
\UseCaseConditions{}
\UseCaseNonFunctionalRequirements{}
\UseCaseScenarios{}
\UseCaseRisks{}
\UseCaseUserInterface{}
\end{UseCaseTemplate}

%%%%%%%%%%%%%%%%%%%%%%%%%%%%%%%%%%%%%%%%%%%%%%%%%%%%%%%%%%%%%%%%%%%%%%%%%%%%%%

\section{Non Functional Requirements}

%Describe the non-functional requirements for the system here, giving a
%rationale (traceable to your requirements gathering) for each.  You
%will need to think about how to group/structure requirements in this
%section.

\begin{itemize}
\item{
The system is expected to be used at students' homes or in the Biology
lab computers, so portability is essential for the system to work to
the clients' expectations.
}
\end{itemize}

%%%%%%%%%%%%%%%%%%%%%%%%%%%%%%%%%%%%%%%%%%%%%%%%%%%%%%%%%%%%%%%%%%%%%%%%%%%%%%
\section{Summary}

Give a (very short) summary of the key aspects of the requirements
specification.

%%%%%%%%%%%%%%%%%%%%%%%%%%%%%%%%%%%%%%%%%%%%%%%%%%%%%%%%%%%%%%%%%%%%%%%%%%%%%%

\appendix

\section{Glossary}


\section{Client Meeting Documentation}
\label{sec:clientMeeetingDocs}

%Any evidence you gathered from stakeholders relevant to your
%requirements description.  You don't need to include everything
%verbatim here, but summary documents, for example, identifying the key
%points you identified (particularly if they relate to requirements
%conflicts) can be useful.

\section{Group Meeting Documentation}

%%%%%%%%%%%%%%%%%%%%%%%%%%%%%%%%%%%%%%%%%%%%%%%%%%%%%%%%%%%%%%%%%%%%%%%%%%%%%%

\end{document}

%%%%%%%%%%%%%%%%%%%%%%%%%%%%%%%%%%%%%%%%%%%%%%%%%%%%%%%%%%%%%%%%%%%%%%%%%%%%%%