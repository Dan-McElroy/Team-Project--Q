\documentclass[A4paper]{article}

\title{Team Project Demonstration Round Table Discussion}
\date{8th February 2013}

\usepackage{fullpage}
\begin{document}
\maketitle
\newcommand{\us}{\textbf{T: }}
\newcommand{\pat}{\textbf{P: }}
\newcommand{\pam}{\textbf{Scott: }}
\newcommand{\nic}{\textbf{Veitch: }}
\textbf{Team Member: }So, can I start by asking how many people got to the melting temperature screen?
\\ 
\textit{*inaudible*}\\
\us So, three to four. Okay.\\  \\
\textit{*feedback requested*}\\
\textbf{Participant:} The repetition rule should tell you exactly how many you missed in a row,
so that you'll know to avoid 4.\\
\us Alright. Any more comments before we get into the specifics?\\
\pat Overall, the problem was that the primer design rules were unbreakable. So, maybe it should 
have more of a guidance system than a rule-based system.\\
\pam Do you think, then, that pass/fail are the wrong words to use?
\pat Yeah, probably.\\
\pat Or just a manual override so that you can accept that you've failed, and move on.\\
\textit{*general agreement*}\\
\pat Maybe implement something like a close fail system?\\ 
\textit{Conversation about food follows.}\\ \\
\textbf{TEMPERATURE PANEL IN USER GUIDE SHOULD BE CHANGED}\\ \\
\us What did you think of the look of the program?\\
\pat It was quite clear to figure out how to get from one point to the next. The appearance wasn't too bad.\\
\nic I thought you could have something like a DNA helix in the corner to make it look... a bit less
clinical. A bit less Computer Science. No offence!\\
\textit{Offence taken.}\\
\pat I thought the interface was very 90s.\\
\us That is not our fault. If you're still using Windows XP, that's not our fault!\\
\textit{Le humour.}\\ \\
\pat I thought the program was very easy - I could come into the lab and do it without any problems, or at
home and I wouldn't need someone there.\\
\pat I thought the double strand view was very helpful, and the reverse button just made things easier.\\
\us As a teaching tool, do you think it was better than a paper version?\\
\textit{*unanimous affirmation*}\\ \\
\pat Why doesn't it find primers for you?\\
\us That's kind of the point.\\
\pat I didn't realise it was supposed to teach me. I thought it was supposed to help me.\\
\pat Yeah, I was using it as a program to, you know, make a primer.\\ \\
\pat The pass/fail breakdown is actually quite good, breaking down what's good about the primer and what's bad.
\pat A bit more detail would be good, actually.\\
\us Yeah, and having "FAIL" all in capital letters is perhaps a bit harsh.\\
\pat It was good that you could amend the sequence on the fly though, if just a little bit of it was wrong you
could go in and fix it.\\
\pat Although, when I kind of passed my reverse primer, it disappeared out the box.\\ 
\pat Would it be possible to have it check one of the primers first before doing the other one?\\ \\
\pam I was thinking, maybe after the temperature panel you could have it show just the sequence, in bold, with
the primers at either end? Just the sequence with the actual primers, with a back button in case you want to 
change it.\\ \\
\pat Copy-pasting was all keyboard shortcuts. I don't use keyboard shortcuts. 

\end{document}

